
\vspace{-0.8em}
\section{The PTA-DTA Problem}
\vspace{-0.8em}

In this section, we define the problem of model-checking PTAs against DTA-specifications.
The problem takes a PTA and a DTA as input, and computes the probability that infinite paths under the PTA are accepted by the DTA.
Informally, the DTA encodes the linear-time property by judging whether an infinite path is accepted or not through the external behaviour of the path,
thus the problem is to compute the probability that the external behaviour of PTA meets the criterion specified by the DTA.
In practice, the DTA is often used to capture all good (or bad) behaviours, so the problem can be treated as a task to evaluated to what extent the PTA behaves in a good (or bad) way.

Below we fix a well-formed PTA $\pta$ taking the form (\ref{eq:pta}) and a DTA $\dta$ taking the form (\ref{eq:dta}) 
with the difference that the set of clocks for $\pta$ (resp. for $\dta$) is denoted by $\clocks_1$ (resp. $\clocks_2$).
W.l.o.g., we assume that $\clocks_1\cap\clocks_2=\emptyset$ and $\alphabet=2^{\ap}$.
%We also fix a reward structure $(\rcum,\rinst)$ for $\pta$.
We first show how an infinite path in $\infpaths{\pta}$ can be interpreted as an infinite word.

\vspace{-0.8em}
\begin{definition}[Infinite Paths as Infinite Words]\label{def:interpretation}
Given an infinite path
\[
\infpath=(\loc_0,\nu_0)a_0(\loc_1,\nu_1)a_1(\loc_2,\nu_2)a_2\dots a_{2n}(\loc_{2n+1},\nu_{2n+1})a_{2n+1}(\loc_{2n+2},\nu_{2n+2})\dots
\]
under $\pta$ (note that $\nu_0=\zero$), the infinite word $\lbfunc(\infpath)$ over $2^{\ap}\cup[0,\infty)$ is defined as 
\[
\lbfunc(\infpath):=a_0\lbfunc(\loc_2)a_2\lbfunc(\loc_4)\dots a_{2n}\lbfunc(\loc_{2n+2}) \dots\enskip.
\]
Recall that $a_{2n}\in [0,\infty)$ and $a_{2n+1}\in\acts$. 
%where $a'_n:= a_n$ if $a_n\in [0,\infty)$ and $a'_n:=\varepsilon$ otherwise ($\varepsilon$ is the empty word).
\end{definition}

\begin{remark}
Informally, the interpretation in Definition~\ref{def:interpretation} works
by (i) dropping (a) the initial location $\loc_0$, (b) all clock valuations $\nu_n$'s, 
(c) all locations $\loc_{2n+1}$'s following a time-elapse,
(d) all internal actions $a_{2n+1}$'s of $\pta$ and (ii) replacing every $\loc_{2n}$ ($n\ge 1$) by $\lbfunc(\loc_{2n})$.
The interpretation captures only external behaviours including time-elapses and labels of locations upon state-change, and discards internal behaviours such as the concrete locations, clock valuations and actions.
Although the interpretation ignores the initial location,
we deal with it in our acceptance condition where the initial location is preprocessed by the DTA.
\end{remark}

\begin{remark}
Our interpretation is different from~\cite{DBLP:journals/tse/DonatelliHS09,DBLP:journals/corr/abs-1101-3694,DBLP:conf/hybrid/Fu13}.
In the style from~\cite{DBLP:journals/tse/DonatelliHS09,DBLP:journals/corr/abs-1101-3694,DBLP:conf/hybrid/Fu13}, an infinite path
$(\loc_0,\nu_0)a_0(\loc_1,\nu_1)a_1\dots$ is interpreted as $a_0 \lbfunc(\loc_0)a_2 \lbfunc(\loc_2)\dots$,
reversing the locations and actions/time-elapses.
In contrast, our interpretation follows a natural way that preserves the order of external events in an infinite path.
This advantage allows one to specify DTAs (for linear-time properties) in a straightforward way.
\end{remark}

Based on Definition~\ref{def:interpretation}, we define the finite acceptance condition as follows.
For an infinite path $\infpath=(\loc_0,\nu_0)a_0(\loc_1,\nu_1)a_1\dots$ under $\pta$, we denote by $\initloc{\infpath}$ the initial location $\loc_0$.

\vspace{-0.8em}
\begin{definition}[Path Acceptance]\label{def:fnacc}
An infinite path $\infpath$ under $\pta$ is \emph{infinitely accepted} by $\dta$ w.r.t initial configuration $(\dtloc,\nu)$ and a Rabin acceptance condition $\rabin$ if the infinite word $\lbfunc(\infpath)$ is accepted by $\dta$ w.r.t $\left(\trfunc\left((\dtloc,\nu), \lbfunc(\initloc{\infpath})\right),\zero\right)$ and $\rabin$.
\end{definition}

%\begin{definition}[Rabin Acceptance]\label{def:infacc}
%Let $\Gamma=\{(E_i,F_i)\}_{i\in I}$ be a finite collection of set pairs indexed by $I$ such that $E_i,F_i\subseteq\cstates$ for all $i\in I$.
%An infinite path $\infpath$ under $\pta$ is \emph{infinitely accepted} by $\dta$ w.r.t initial configuration $(\dtloc,\nu)$ and $\Gamma$ if the infinite word $\lbfunc(\infpath)$ is accepted by $\dta$ w.r.t $\left(\trfunc\left((\dtloc,\nu), \lbfunc(\initloc{\pi})\right),\zero\right)$ and $\Gamma$.
%\end{definition}

In the definitions above, the initial location omitted in Definition~\ref{def:interpretation} is preprocessed by specifying explicitly that the initial configuration is $\left(\trfunc\left((\dtloc,\nu), \lbfunc(\initloc{\infpath})\right),\zero\right)$.

Now we define the notion of acceptance probabilities over infinite paths under $\pta$.
\vspace{-0.8em}
\begin{definition}[Acceptance Probabilities]
Let $F$ be a Rain acceptance condition.
The probability that $\pta$ \emph{observes} $\dta$ under scheduler $\sigma$, initial mode $\dtloc\in\cstates$ and $F$, denoted by $\pr{\dtloc,F}{\sigma}$, is defined by:
\[
    \pr{\dtloc,\rabin}{\sigma}
        :=
            \probm^{\pta,\sigma}\left(
                % \acc{\pta,\sigma}{\dta,q,F}
                \LangCsAqF
            \right)
\]
% \pr{\dtloc,F}{\sigma}:=\probm^{\pta,\sigma}\left(\acc{\pta,\sigma}{\dta,q,F}\right)
where $\LangCsAqF$ is paths in $\pta$ that falls into the Rabin-accepted language of $\dta$
$$
    \LangCsAqF = \left \{ 
        \infpath \in \infpaths{\pta,\sigma} \mid 
        \infpath
        \mbox{ is accepted by } \dta \mbox{ w.r.t. } (q,\zero) \mbox{ and } \rabin 
    \right\}
$$
% $\acc{\pta,\sigma}{\dta,q,F}:=\left\{\infpath\in\infpaths{\pta,\sigma}\mid \infpath\mbox{ is accepted by }\dta\mbox{ w.r.t. }(q,\zero)\mbox{ and }F\right\}$.
\end{definition}
Again, from the fact that the set $\fnpaths{\pta,\sigma}$ is countably-infinite, 
$\acc{\pta,\sigma}{\dta,q,F}$ is \textcolor[rgb]{1,0,0}{measurable since it can be represent int the 
form of a countable intersect and countable union of some cylinder sets}.

Now the {\sc PTA-DTA} problem is as follows.

\begin{compactitem}
\item {\bf Input:} a well-formed PTA $\pta$, a DTA $\dta$, an initial mode $\dtloc$ and a subset $\rabin$ of modes;
\item {\bf Output:} $\inf_\sigma \pr{\dtloc,\rabin}{\sigma}$ and $\sup_\sigma  \pr{\dtloc,\rabin}{\sigma}$, where $\sigma$ ranges over all time-divergent schedulers.
\end{compactitem}

%In this paper, we also consider a variant of {\sc PTA-DTA} which incorporates rewards.
%We first migrate the notion of cumulative reward to DTAs.

%\begin{definition}[Cumulative Reward Until Acceptance]
%Let $\Gamma$ be a subset of $\locs$ and $(\rcum,\rinst)$ be a reward structure for $\pta$.
%The random variable $\accum{F}$ over infinite paths is defined as follows:
%for any infinite path $\infpath$ and its associated run
%$\run{\dta}{\dtloc,\nu}{\lbfunc(\infpath)}=\{(\dtloc_n,\nu_n,a_n)\}_{n\in\Nset_0}$,
%\[
%\accum{F}(\infpath):=
%\begin{cases}
%\sum_{k=0}^{n^*-1} \ronestep(\loc_k, a_k) & \mbox{if }\{n\mid \dtloc_n\in F\}\ne\emptyset\mbox{ and }n^*=\min\{n\mid \dtloc_n\in F\}\\
%\infty & \mbox{otherwise}
%\end{cases}\enskip.
%\]
%Then the \emph{expected cumulative reward} $\rd{\dtloc}{\sigma,\Gamma}$ is defined as $\rd{\dtloc}{\sigma,\Gamma}:=\expv^{\pta,\sigma}(\accum{F})$, where $\expv^{\pta,\sigma}$ is the expectation operator for $\probm^{\pta,\sigma}$\enskip.
%\end{definition}

%Then we integrate reward into the {\sc PTA-DTA} problem. We call it {\sc REWARD-PTA-DTA} problem.

%\begin{compactitem}
%\item {\bf Input:} a PTA $\pta$, a DTA $\dta$, a reward structure $(\rcum,\rinst)$, an initial mode $\dtloc$ and a $\Gamma$ given as in either Definition~\ref{def:fnacc} or Definition~\ref{def:infacc} such that $\inf_\sigma \pr{\dtloc,\Gamma}{\sigma}=1$;
%\item {\bf Output:} $\inf_\sigma \rd{\dtloc}{\sigma,\Gamma}$ and $\sup_\sigma  \rd{\dtloc}{\sigma,\Gamma}$, where $\sigma$ ranges over all schedulers.
%\end{compactitem}

