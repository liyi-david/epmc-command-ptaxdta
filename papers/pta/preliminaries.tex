\vspace{-0.8em}
\section{Preliminaries}
\vspace{-0.8em}

In the whole paper, we denote by $\Nset$, $\Nset_0$, $\Zset$, and $\Rset$ the sets of all positive
integers, non-negative integers, integers, and real numbers, respectively.

For an infinite string $w$, $\infset{w}$ is the set of symbols that occur infinitely many times in $w$.
\vspace{-0.8em}
\subsection{Clock Valuations, Clock Constraints and Clock Equivalences}
\vspace{-0.8em}

In this part, we fix a finite set $\clocks$ of \emph{clocks}.

\noindent{\em Clock Valuations.} Let $\clocks$ be a finite set of \emph{clocks}.
A \emph{clock valuation} is a function $\nu:\clocks\rightarrow [0,\infty)$. The set of clock valuations
is denoted by $\val{\clocks}$.
Given a clock valuation $\nu$, a subset $X\subseteq\clocks$ of clocks and a non-negative real number $t$, we let (i) $\reset{\nu}{X}$ be the clock valuation such that $\reset{\nu}{X}(x)=0$ for $x\in X$ and $\reset{\nu}{X}(x)=\nu(x)$ otherwise, and (ii) $\add{\nu}{t}$ be the clock valuation such that $(\add{\nu}{t})(x)=\nu(x)+t$ for all $x\in\clocks$.
Moreover, we denote by $\zero$ the clock valuation such that $\zero(x)=0$ for all $x\in\clocks$.

\noindent{\em Clock Constraints.} The set of \emph{clock constraints} $\clcons{\clocks}$ over $\clocks$ is generated by the following grammar:
\[
\phi:=~\true~\mid~x\le d~\mid ~c\le x~\mid~x+c\le y+d~\mid ~\neg\phi~\mid~\phi\wedge\phi
\]
where $x,y\in\clocks$ and $c,d\in\Nset_0$.
We write $\false$ for a short hand of $\neg\true$.
The satisfaction relation $\models$ between valuations $\nu$ and clock constraints $\phi$ is defined through substituting every $x\in\clocks$ appearing in $\phi$ by $\nu(x)$ and standard semantics for logical connectives.
For a given clock constraint $\phi$, we denote by $\sat{\phi}$ the set of all clock valuations that satisfy $\phi$.

\noindent{\em Clock Equivalence.} Consider a nonnegative integer $N$ which acts a threshold for relevant clock values: values held by clocks are treated the same if they exceed $N$.
With such a fixed $N$, the standard notion of clock equivalence (see~\cite{DBLP:journals/tcs/AlurD94}) is an equivalence relation $\sim_N$ over $\val{\clocks}$ as follows: for any two clock valuations $\nu,\nu'$, $\nu\sim_N\nu'$ iff the following conditions hold:
\begin{compactitem}
\item for all $x\in\clocks$, $\nu(x)>N$ iff $\nu'(x)> N$;
\item for all $x\in\clocks$, if $\nu(x)\le N$ then (i) $\intp{\nu(x)}=\intp{\nu'(x)}$ and (ii) $\fracp{\nu(x)}>0$ iff $\fracp{\nu'(x)}>0$;
\item for all $x,y\in\clocks$, if $\nu(x),\nu(y)\le N$ then $\fracp{\nu(x)}\Join \fracp{\nu(y)}$ iff $\fracp{\nu'(x)}\Join \fracp{\nu'(y)}$ for all $\Join\in\{<,=,>\}$.
\end{compactitem}
Equivalence classes of $\sim_N$ are conventionally called \emph{regions}. The equivalence class that contains a given clock valuation $\nu$ is conventionally denoted by $\evclass{\nu}_\sim$.
\vspace{-0.8em}
\subsection{Probabilistic Timed Automata}
\vspace{-0.8em}
To introduce the notion of probabilistic timed automata (PTAs), we first define the notion of discrete probability distributions.

\noindent{\em Discrete Probability Distributions.} A \emph{discrete probability distribution} over a countable non-empty set $U$ is a function $q:U\rightarrow[0,1]$ such that $\sum_{z\in U}q(z)=1$.
The \emph{support} of $q$ is defined as $\supp{q}:=\{z\in U\mid q(z)>0\}$.
The set of discrete probability distributions over $U$ is denoted by $\dist{U}$.
For $u \in U$, let $\mu(u)$ be the \emph{point distribution at $u$} which assigns probability 1 to $u$.

\vspace{-0.8em}
\begin{definition}[Probabilistic Timed Automata (PTAs)~\cite{DBLP:journals/fmsd/NormanPS13}]
A \emph{probabilistic timed automaton} (PTA) $\pta$ is a tuple
\begin{equation}\label{eq:pta}
\pta=\left(\locs, \loc^*, \clocks, \acts, \inv, \enab,  \prob, \lbfunc\right)
\end{equation}
where
\begin{compactitem}
\item $\locs$ is a finite set of \emph{locations} and $\loc^*\in\locs$ is the \emph{initial} location;
\item $\clocks$ is a finite set of \emph{clocks};
\item $\acts$ is a finite set of \emph{actions};
\item $\inv:\locs\rightarrow\clcons{\clocks}$ is
%a function which assigns to every location
an \emph{invariant condition};
\item $\enab:\locs\times\acts\rightarrow\clcons{\clocks}$ is an \emph{enabling condition};
\item $\prob:\locs\times\acts\rightarrow\dist{2^{\clocks}\times\locs}$ is a \emph{probabilistic transition function};
\item $\ap$ is a finite set of \emph{atomic propositions} and $\lbfunc:\locs\rightarrow 2^{\ap}$ is a \emph{labelling function}.
\end{compactitem}
\end{definition}
\vspace{-0.8em}

W.l.o.g, we assume that both $\acts$ and $\ap$ is disjoint from $[0,\infty)$. Below we fix a PTA $\pta$ in the form (\ref{eq:pta}). The semantics of PTAs is as follows.

\noindent{\em States and Transition Relation.}
A \emph{state} of $\pta$ is a pair $(\loc, \nu)$ in $\locs\times\val{\clocks}$ such that $\nu\models \inv(\loc)$.
The set of all states is denoted by $\states_\pta$.
The \emph{transition relation} $\trans$ consists of all triples $((\loc,\nu),a,(\loc',\nu'))$ satisfying that
\begin{compactitem}
\item $(\loc,\nu), (\loc',\nu')$ are states and $a\in\acts\cup [0,\infty)$;
\item if $a\in [0,\infty)$ then $\nu+\tau\models \inv(\loc)$ for all $\tau\in [0, a]$ and $(\loc',\nu')=(\loc,\nu+a)$;
\item if $a\in\acts$ then $\nu\models\enab(\loc,a)$ and there exists a pair $(X, \loc'')\in\supp{\prob(\loc,a)}$ such that $(\loc',\nu')=(\loc'',\reset{\nu}{X})$.
\end{compactitem}
By convention, we write $\tran{s}{a}{s'}$ instead of $(s,a,s')\in\trans$.
We omit the subscript `$\pta$' in `$\states_\pta$' if the underlying context is clear.
The \emph{probability transition kernel} $\probk$ is the function $\probk:\states\times\acts\times\states\rightarrow[0,1]$ such that
\[
\probk((\loc,\nu),a,(\loc',\nu'))=\begin{cases}
1 & \mbox{if }\tran{(\loc,\nu)}{a}{(\loc',\nu')}\mbox{ and } a\in [0,\infty)\\
\sum_{Y\in B}\prob(\loc,a)(Y,\loc') & \mbox{if }\tran{(\loc,\nu)}{a}{(\loc',\nu')}\mbox{ and } a\in\acts \\
0 & \mbox{otherwise}
\end{cases}
\]
where $B:=\{X\subseteq\clocks\mid \nu'=\reset{\nu}{X}\}$.

\noindent{\em Well-formedness.} We say that $\pta$ is \emph{well-formed} if for every state $(\loc,\nu)$ and action $a\in\acts$ such that $\nu\models\enab(\loc,a)$ and for every $(X,\loc')\in \supp{\prob(\loc,a)}$, one has that $\reset{\nu}{X}\models\inv(\loc')$.
The well-formedness is to ensure that when an action is enabled, the next state after taking this action will always be legal. In the rest of the paper, we always assume that the underlying PTA is well-formed. PTAs that are not well-formed can be repaired to satisfy the well-formedness condition~\cite{DBLP:journals/iandc/KwiatkowskaNSW07}.

\noindent{\em Paths.}
A \emph{finite path} $\fnpath$ (under $\pta$) is a finite sequence
\[
\left\langle s_0,a_0,s_1,\dots,a_{n-1},s_n\right\rangle~~(n\ge 0)
\]
in $\states\times{\left((\acts\cup[0,\infty))\times \states\right)}^*$
such that (i) $s_0=(\loc^*,\zero)$,
(ii) $a_{2k}\in [0,\infty)$ for all integers $0\le k\le \frac{n}{2}$, (iii) $a_{2k+1}\in \acts$ for all integers $0\le k\le \frac{n-1}{2}$ and (iv) for all $0\le k\le n-1$, $\tran{s_k}{a_k}{s_{k+1}}$.
The length of $\fnpath$ is n, denoted by len($\fnpath$)
An \emph{infinite path} (under $\pta$) is an infinite sequence
\[
\left\langle s_0,a_0,s_1,a_1,\dots\right\rangle
\]
in ${\left(\states\times(\acts\cup[0,\infty))\right)}^\omega$
such that for all $n\in\Nset_0$, $\left\langle s_0,a_0,\dots,a_{n-1},s_n\right\rangle$ is a finite path.
The set of finite (resp. infinite) paths  under $\pta$ is denoted by $\fnpaths{\pta}$ (resp. $\infpaths{\pta}$).



\noindent{\em Schedulers.}
A \emph{scheduler} (or \emph{adversary}) is a function $\sigma$ from the set of finite paths into $\acts\cup [0,\infty)$ such that for all finite paths $\fnpath=s_0a_0\dots s_n$,
(i) $\sigma(\fnpath)\in\acts$ if $n$ is odd, (ii) $\sigma(\fnpath)\in  [0,\infty)$ if $n$ is even, and (iii)
there exists a state $s'$ such that $\tran{s_n}{\sigma(\fnpath)}{s'}$.
A finite path $\fnpath=s_0a_0\dots s_n$ is said to \emph{follow} a scheduler $\sigma$ if for all $0\le m\le n$, $a_m=\sigma\left(s_0a_0\dots s_m\right)$.
Likewise, an infinite path $s_0a_0s_1a_1\dots$ \emph{follows} a scheduler $\sigma$ if for all $n\in\Nset_0$, $a_n=\sigma\left(s_0a_0\dots s_n\right)$.
The set of finite (resp. infinite) paths following a scheduler $\sigma$ is denoted by $\fnpaths{\pta,\sigma}$ (resp. $\infpaths{\pta,\sigma}$).
We note that the set $\fnpaths{\pta,\sigma}$ is countably-infinite from definition.

\noindent{\em Probability Spaces under Schedulers.}
Let $\sigma$ be any scheduler for $\pta$.
The probability space for $\pta$ w.r.t $\sigma$ is defined as $(\Omega^{\pta,\sigma}, \mathcal{F}^{\pta,\sigma}, \probm^{\pta,\sigma})$ where $\Omega^{\pta,\sigma}:=\infpaths{\pta,\sigma}$, $\mathcal{F}^{\pta,\sigma}$ is the smallest $\sigma$-algebra generated by all cylinder sets induced by finite paths (a finite path $\fnpath$ induces the cylinder set $\cyl(\fnpath)$ of all infinite paths in $\infpaths{\pta,\sigma}$ with $\fnpath$ being their (common) prefix)
and $\probm^{\pta,\sigma}$ is the unique probability measure such that for all finite paths $\fnpath=s_0a_0\dots a_{n-1}s_n$ in $\fnpaths{\pta,\sigma}$,
$\probm^{\pta,\sigma}(\cyl(\fnpath))=\prod_{k=0}^{n-1} \probk(s_k, \sigma(s_0a_0\dots a_{k-1}s_k), s_{k+1})$.
Intuitively, the probability space under $\sigma$ is induced by a Markov chain where the state space is $\fnpaths{\pta,\sigma}$ and the one-step probability transition matrix is determined by $\probk$ and $\sigma$.

\noindent{\em Zenoness and Time-Divergent Schedulers.}
An infinite path $\infpath=s_0a_0s_1a_1\dots$ is \emph{zeno} if $\sum_{n=0} d_n=\infty$, where $d_n:=a_n$ if $a_n\in [0,\infty)$ and $d_n:=0$ otherwise.
Then a scheduler $\sigma$ is \emph{time divergent} if $\probm^{\pta,\sigma}(\{\pi\mid\pi\mbox{ is zeno}\})=0$.
In the rest of the paper, we only consider time-divergent schedulers.
The purpose to restrict to time-divergent schedulers is to eliminate non-realistic zeno behaviours such as performing infinitely many actions within a bounded amount of time.

\noindent{\em Reachability.} An infinite path $\infpath=(\loc_0,\nu_0)a_0(\loc_1,\nu_1)a_1\dots$ is said to \emph{visit} a subset $U\subseteq\locs$ of locations \emph{eventually} if there exists $n\in\Nset_0$ such that $\loc_n\in U$. The set of infinite paths in $\infpaths{\pta,\sigma}$ that visit $U$ eventually is denoted by $\omgpaths{\pta,\sigma}{U}$.
From the fact that the set $\fnpaths{\pta,\sigma}$ is countably-infinite, $\omgpaths{\pta,\sigma}{U}$ is measurable since it is a countable union of cylinder sets.

%In the following, we introduce the notion of \emph{rewards} (or \emph{costs}) over PTAs.

%\begin{definition}[Rewards for PTAs]
%A \emph{reward structure} for the PTA $\pta$ is a pair of functions $(\rcum,\rinst)$ such that $\rcum: \locs\rightarrow [0,\infty)$ is a function assigning to every location a rate at which rewards are accumulated as time elapses in that location, while $\rinst: \locs\times\acts\rightarrow [0,\infty)$ assigns to each location-action pair an instantaneous reward when the action is taken at that location).
%The one-step reward function $\ronestep:\locs\times\acts\rightarrow [0,\infty)$ is defined by:
%\[
%\ronestep(\loc, a)=
%\begin{cases}
%\rinst(\loc, a) & \mbox{if }a\in\acts \\
%a\cdot \rcum(\loc) & \mbox{ if }a\in [0,\infty)
%\end{cases}\enskip.
%\]
%\end{definition}

%In this paper, we focus on \emph{cumulative reward} over paths.

%\noindent{\em Cumulative Reward.} Let $U\subseteq\locs$. We define the \emph{cumulative reward function} $\accum{U}$ (until $U$ is reached) on $\infpaths{\pta}$ such that for any infinite path $\infpath=(\loc_0,\nu_0)a_0(\loc_1,\nu_1)a_1\dots$,
%\[
%\accum{U}(\infpath)=
%\begin{cases}
%\sum_{k=0}^{n^*-1} \ronestep(\loc_k, a_k) & \mbox{if }\{n\mid \loc_n\in U\}\ne\emptyset\mbox{ and }n^*=\min\{n\mid \loc_n\in U\}\\
%\infty & \mbox{otherwise}
%\end{cases}\enskip.
%\]

%\noindent{\em Repeated Reachability.} Let $\sigma$ be any scheduler for $\pta$.
%Consider a subset $U$ of $\locs$. An infinite path $(\loc_0,\nu_0)a_0(\loc_1,\nu_1)a_1\dots$ is said to \emph{visit} $U$ \emph{infinitely often} if the set $\{n\in\Nset_0\mid \loc_n\in U\}$ is infinite.
%The set of all infinite paths visiting $U$ infinitely often is denoted by $\omgpaths{\pta,\sigma}{U}$.
\vspace{-0.8em}
\subsection{Deterministic Timed Automata}
\vspace{-0.8em}
\begin{definition}[Deterministic Timed Automata (DTAs)~\cite{DBLP:journals/tse/DonatelliHS09,DBLP:journals/corr/abs-1101-3694,DBLP:conf/hybrid/Fu13}]
A \emph{deterministic timed automaton} (DTA) $\dta$ is a tuple
\begin{equation}\label{eq:dta}
\dta=(\cstates,\alphabet,\clocks,\rules)
\end{equation}
where
\begin{compactitem}
\item $\cstates$ is a finite set of \emph{modes}; %and $\dtloc^*$ is the \emph{initial} mode;
\item $\alphabet$ is a finite \emph{alphabet} of \emph{symbols} disjoint from $[0,\infty)$;
\item $\mathcal{X}$ is a finite set of \emph{clocks};
\item $\rules\subseteq \cstates\times\alphabet\times\clcons{\clocks}\times 2^{\clocks}\times \cstates$ is a finite set of \emph{rules} such that
\begin{compactenum}
\item ({\em determinism}): whenever $(\dtloc_1,b_1,\phi_1,X_1,\dtloc'_1),(\dtloc_2,b_2,\phi_2,X_2,\dtloc'_2)\in\rules$, if $(\dtloc_1,b_1)=(\dtloc_2,b_2)$ and $\sat{\phi_1}\cap \sat{\phi_2}\ne\emptyset$ then $(\phi_1,X_1,\dtloc'_1)=(\phi_2,X_2,\dtloc'_2)$;
\item ({\em totality}): for all $(q,b)\in \cstates\times\Sigma$ and $\nu\in\val{\clocks}$, there exists $(q,b,\phi,X,q')\in\Delta$ such that $\nu\models \phi$.
\end{compactenum}
%\item $\fstates\subseteq\cstates$ is a set of \emph{final modes}.
\end{compactitem}
\end{definition}

Below we fix a DTA $\dta$ in the form~(\ref{eq:dta}). Given $q\in\cstates$, $\nu\in\val{\clocks}$ and $b\in\Sigma$, the triple $(\dtphi{q}{b}{\nu},\dtx{q}{b}{\nu},\dtq{q}{b}{\nu})\in\clcons{\clocks}\times 2^{\clocks}\times\cstates$ are determined such that $\left(\dtloc,b,\dtphi{q}{b}{\nu},\dtx{q}{b}{\nu},\dtq{q}{b}{\nu}\right)\in\rules$ is the unique rule satisfying $\nu\models \dtphi{q}{b}{\nu}$.
We illustrate the semantics of DTAs as follows.

{\em Configurations and One-Step Transition Function.}
A \emph{configuration} of $\dta$ is a pair $(\dtloc,\nu)$, where $\dtloc\in \cstates$ and $\nu\in\val{\clocks}$.
The \emph{one-step transition function}
\[
\kappa:(\cstates\times\val{\clocks})\times (\alphabet\cup [0,\infty))\rightarrow \cstates\times\val{\clocks}
\]
is defined by: $\trfunc((\dtloc,\nu),a):=\left(\dtq{q}{a}{\nu},\reset{\nu}{\dtx{q}{a}{\nu}}\right)$ for $a\in\alphabet$; $\trfunc((\dtloc,\nu),a):=(\dtloc,\nu+a)$ for $a\in [0,\infty)$.
For the sake of convenience, we write $\dtatr{(\dtloc,\nu)}{a}{(\dtloc',\nu')}$
instead of $\kappa((\dtloc,\nu),a)=(\dtloc',\nu')$.

{\em Infinite Words and Runs.}
An \emph{infinite word} is an infinite sequence $\{a_n\}_{n\in\Nset_0}$ such that $a_n\in \acts\cup [0,\infty)$ for all $n$.
%
% ACT ????????? sigma
%
The \emph{run} of $\dta$ on an infinite word $w=\{a_n\}_{n\in\Nset_0}$
%over $\acts\cup [0,\infty)$ (i.e., $a_n\in \acts\cup [0,\infty)$ for all $n$)
with \emph{initial configuration} $(\dtloc,\nu)$, denoted by $\run{\dta}{\dtloc,\nu}{w}$, is the unique infinite
sequence $\{\left(\dtloc_n,\nu_n,a_n\right)\}_{n\in\Nset_0}$
which satisfies that $(\dtloc_0,\nu_0)=(\dtloc,\nu)$ and $\dtatr{(\dtloc_n,\nu_n)}{a_n}{(\dtloc_{n+1},\nu_{n+1})}$
for all $n\in\Nset_0$.
%%%%%
The trajectory of $\run{\dta}{\dtloc,\nu}{w}$, an infinite string over $\cstates$,
is define as follow $\traj{ \run{\dta}{\dtloc,\nu}{w} } := q_0 q_1 \dots$
%%%%%

Now we illustrate the acceptance condition for DTAs. In this paper, we focus on infinite acceptance condition. Finite case is trivial and we also support it in our tool.
% \vspace{-0.8em}
% \begin{definition}[Finite Acceptance Criterion]
% Let $F\subseteq\cstates$ be a set of \emph{final} modes.
% An infinite word $w$ is \emph{(finitely) accepted} by $\dta$ w.r.t the \emph{initial configuration} $(\dtloc,\nu)$ and $F$ if $\run{\dta}{\dtloc,\nu}{w}=\{(\dtloc_n,\nu_n,a_n)\}_{n\in\Nset_0}$ satisfies that $\dtloc_n\in F$ for
% some $n\in\Nset_0$.
% \end{definition}

%%%%%
\vspace{-0.8em}
\begin{definition}[Rabin Acceptance Criterion]
An Rabin acceptance condition is 
$
    \rabin 
        = \left\{ 
            (H_1,K_1 ), 
            \dots ,
            (H_n,K_n) 
        \right\} 
$ .
A set $\cstates' \subseteq \cstates $ is called Rabin accepting by $\rabin$ 
if there exists $ 1 \leq i \leq n$ such that $ \cstates' \cap H_i= \emptyset $ 
and $ \cstates' \cap K_i \neq \emptyset $. An infinite word $w$ is accepted  
$\dta$ with \emph{initial configuration} $(\dtloc,\nu)$ and acceptance $\rabin$ iff    
$
    \mbox{  }
    \infset{ 
        \traj{ 
            \run{\dta}{(\dtloc,\nu)}{w} 
        }
    }
$ is Rabin accepting by $\rabin$.
\end{definition}
%%%%%

%Then we present Rabin acceptance condition which requires an infinite word to visit some modes only finitely often and some other modes infinitely often.

%\begin{definition}[Rabin Acceptance Condition]
%Let $\Gamma=\{(E_i,F_i)\}_{i\in I}$ be a finite collection of set pairs indexed by $I$ such that $E_i,F_i\subseteq\cstates$ for all $i\in I$.
%An infinite word $w$ is \emph{infinitely accepted} by $\dta$ w.r.t the \emph{initial configuration} $(\dtloc,\nu)$ and $\Gamma$ if $\run{\dta}{\dtloc,\nu}{w}=\{(\dtloc_n,\nu_n,a_n)\}_{n\in\Nset_0}$ satisfies that there exists $i\in I$ such that $\cstates'\cap E_i=\emptyset$ and $\cstates'\cap F_i\ne\emptyset$, where
%$\cstates':=\{\dtloc\in\cstates\mid \dtloc=\dtloc_n \mbox{ for infinitely many }n\mbox{'s}\}$.
%\end{definition}
