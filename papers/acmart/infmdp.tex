% \vspace{-0.8em}
\section{Infinite-State-MDP Construction}
% \vspace{-0.8em}

Now we present the finite acceptance of nodeterministic timed automata for PTAs.
\vspace{-0.8em}
\begin{definition}[Finite Acceptance Criterion]
Let $F\subseteq\cstates$ be a set of \emph{final} modes.
An infinite word $w$ is \emph{finitely accepted} by $\dta$ w.r.t the \emph{initial configuration} $(\dtloc,\nu)$ and $F$ if $\run{\dta}{\dtloc,\nu}{w}=\{(\dtloc_n,\nu_n,a_n)\}_{n\in\Nset_0}$ satisfies that $\dtloc_n\in F$ for
some $n\in\Nset_0$.
\end{definition}

\begin{definition}[Path Acceptance]
An infinite path $\infpath$ under $\pta$ is \emph{finitely accepted} by $\tra$ w.r.t 
initial configuration $(\dtloc,\nu)$, if the infinite word $\lbfunc(\infpath)$ is finitely
accepted by $\tra$ w.r.t 
$
\left(\trfunc\left((\dtloc,\nu), \lbfunc(\initloc{\infpath})\right),\zero\right)
$.
\end{definition}
% 
% \begin{definition}[floor Operator]
% For a region $\reg$ with clocks $\clocks$, $\floor{\reg} : \clocks \rightarrow \Nset$ 
% is defined as $ \floor{\reg}(x) = t $ where $t$ is the unique integer s.t. 
% $\reg \models t \le x < t+1$
% \end{definition}
% 
Below we fix a well-formed PTA $\pta$ taking the form (\ref{eq:pta}) and a NTA $\nta$ taking the form (\ref{eq:nta}) with the difference that the set of clocks for $\pta$ (resp. for $\nta$) is denoted by $\clocksX$ (resp. $\clocksY$).
W.l.o.g., we assume that $\clocksX \cap \clocksY = \emptyset$ and $\alphabet=2^{\ap}$.

Let PTA be $\pta$ with the set $\clocksX$ and the NTA be $\nta$ with the set $\clocksY$.

The transformation to MDP is as follows.

Let $\clocksY$ be a fixed finite set of clocks. We use integer Subscript denote a set of 
new clocks. Formally
$
    \clocksY_k = \left \{
        \left (
            t,y
        \right ) \in \Nset \times \clocksY
        \mid
        t = k
    \right \}
$ for $ k > 0 $.
For convenience, we use $ \clocksY_0 $ denote $ \clocksX $.

And $\reg^{\clocksY_k}$ is a region for $\clocksY_k$.
\begin{definition}[Product Construction (Infinite-State-MDP)]
The \emph{product MDP} $\productmdp{\pta}{\dta_q}$ between $\pta$ and $\dta$ with initial mode $\dtloc$ is defined as the PTA

\newcommand{\clocksN}{
    \clocksX \cup \left(
        \bigcup_{k=1}^{n} \clocksY_k
    \right )
}

The transformation to MDP is follows. A state in $\productmdp{\pta}{\dta_q}$ 
is of the form 

\begin{equation}\label{eq:state0}
    \left (
        \loc
        ,
        \left (
            \dtloc_1,
            \cdots,
            \dtloc_n
        \right )
        ,
        \clocksALL
        ,
        \reg
    \right )
\end{equation}

where $n$ is an unbounded natural number, $\loc$ (w.r.t $\dtloc_i$) is a location in $\pta$ 
(w.r.t a mod in $\nta$) and $\reg$ is a region with clock names being $\clocksALL$.
The intuition is that 
$
\pair
    {\loc}
    {\project{\reg}{\clocksX}}
$ 
reflects the region for $\pta$,
$ 
\left (
    \pair
        {\dtloc_1}
        {\project{\reg}{\clocksY_1}}
    \cdots,
    \pair
        {\dtloc_n}
        {\project{\reg}{\clocksY_n}}
\right )
$
reflects a power set for $\nta$.
\end{definition}
% 
% \begin{definition}[Consistency]
% A clock valuation $\nu$ is consistent with a state $\mdploc$ in the form of ~(\ref{eq:state0}), 
% denoted by $ \nu \in  \mdploc$
% iff $\nu \in \clocksN$, $ \nu \downarrow \clocksY_i \in \reg^{\clocksY_i} $ and 
% $\fracp{\nu(x)} < \fracp{\nu(y)}$ iff $x \subseteq y$ for all $x,y \in \clocksN$.
% \end{definition}
% 
\begin{definition}[Rename function]
Let $\clocksX$ and $\clocksY$ be two sets of clocks, $\nu$ is a clock valuation on $\clocksX$ and
$ f:\clocksX \leftrightarrow \clocksY $ is a rename function then $ \nu[f]=\nu \circ f^{-1} $.
% 
\end{definition}
\begin{lemma}
Let $\reg$ is a region with clock names $\clocksX$ and $X \subseteq \clocksX$,
$
    \project
        {\reg}
        {X}
$
is a region with clock names $X$.
\end{lemma}
\begin{definition}[Time successor]
A state
\begin{align*}
    \mdploc'
    =
    \left (
        \loc
        ,
        \left (
            \dtloc_1,
            \cdots,
            \dtloc_n
        \right )
        ,
        \clocksALL
        ,
        \reg'
    \right )
\end{align*}
is a time successor of $\mdploc$ in the form of ~(\ref{eq:state0}) where either 
\begin{compactitem}
    \item 
        $\mdploc = \mdploc'$ if 
        $
            \forall \nu \in \reg, t \in \Rset_{>0} : \nu + t \in \reg'
        $ or
    \item 
        $\reg'$ is another unique region if there exist a $ \nu \in \reg $ s.t.
        \begin{align*}
            \exists t \in \Rset_{\ge0} : (
                \nu + t \in \reg' 
                \land
                \forall t' \in [0,t] : \\ \left (
                    \left (
                        \nu + t' \in \reg \cup \reg'
                    \right )
                    \land
                    \nu + t' \models \inv(l)
                \right )
            )
        \end{align*}
\end{compactitem}
\end{definition}

\begin{definition}[Transition relation]

The \emph{transition relation} $\trans$ is the smallest relation such that the following two inference rules are satisfied : \\
(Delay)
$
    \begin{array}{c}
        \mdploc' \mbox{ is the time successor of } \mdploc \\
        \hline
        \tran
            {\mdploc}
            {\tau}
            {\mu_{\mdploc'}}
    \end{array}
$
\\
(Jump)
$
    \begin{array}{ccc}
        \nu \in \reg
        &
        \tran
            {
                \pair
                    {\loc}
                    {\project{\nu}{\clocksX}}
            }
            {a}
            {\mu}
        &
        \project{\nu}{\clocksX} \models \penab{\loc}{a}
        \\
        \hline
        &
        \tran
            {\mdploc}
            {a}
            {\mu^*}
        &
    \end{array}
$
\\
where, let 
\\
$$
\mdploc' =  \left (
    \loc'
    ,
    \left (
        \dtloc_{1_0}
        \cdots,
        % \dtloc_{1_{k_1}}
        % \cdots,
        \dtloc_{i_0}
        \cdots,
        \dtloc_{i_{k_i}}
        % \dtloc_{n_0}
        \cdots,
        \dtloc_{n_{k_n}}
        % \cdots,
    \right )
    ,
    \clocksX \cup \left(
        \bigcup_{i=1}^{n'} \bigcup_{j=0}^{k_i} \clocksY_{i_j}
    \right )
    ,
    \reg'
\right )
$$

\begin{align*}
    \mu^* \left (
       \mdploc'
    \right )
    = 
    \begin{cases}
        \mu(X,\loc')
        &
        \dag
        \\
        0
        & 
        \mbox{  otherwise }
    \end{cases}
\end{align*} 

The none zero case hold
$
    \mbox{ if } (
        \project
            {\reg'}
            {\clocksX}) 
        = 
        \evclass{
            \project
                {\nu}
                {\clocksX}
            [X := 0 ]
        }_\sim 
$, there exists
$
    \left (
        \dtloc_i,
        \lbfunc\left(\loc'\right),
        \phi,
        Y_{i_j},
        \dtloc_{i_j}
    \right) \in \rules
$ 
such that \\
$
    \project
        {\reg'}
        {\clocksY_{i_j}}
    = 
        \evclass{ \left (
                \project
                    {\nu} 
                    {\clocksY_{i}}
            \right) 
            [ Y_{i_j} := 0 ] 
            [ y \mapsto <i_j,y> ]
        }_\sim 
$ 
and 
$    
    \project
        {\reg}
        {\clocksY_{i}}
    \subseteq \sat{\phi}
$,
$ k_i $ is the number of successors of $ \dtloc_i $. 

\end{definition}
