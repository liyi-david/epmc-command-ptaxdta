% \vspace{-0.8em}
\section{Infinite-State-MDP Construction}
% \vspace{-0.8em}
$\left({\locs}_\otimes, \loc^*_\otimes, \clocks_\otimes, \acts_\otimes, \inv_\otimes, \enab_\otimes,  \prob_\otimes, \lbfunc_\otimes\right)$, where:
\begin{compactitem}
\item $\locs_\otimes:=\locs\times \cstates$;
\item $\loc^*_\otimes:=(\loc^*, q^\star)$ where $q^\star$ is the unique mode\\
    such that $\trfunc\left((\dtloc,\zero), \lbfunc(\loc^*)\right)=(q^\star,\zero)$;
    %and ${\left[\zero\right]}_\sim$ is the region which contains the sole element $\zero$;
\item $\clocks_\otimes:=\clocks_1\cup\clocks_2$;
\item $\acts_\otimes:=\acts\times\regions$; %$\acts_\otimes:=\acts\times\rules$;
\item $\inv_\otimes(\loc,\dtloc):=\inv(\loc)$ for all $(\loc,\dtloc)\in \locs_\otimes$;
\item $\enab_\otimes\left((\loc,\dtloc), (a,R)\right):=\enab(\loc,a)\wedge \phi_R$ for all $(\loc,\dtloc)\in \locs_\otimes$, where $\phi_R$ is any clock constraint such that $\sat{\phi_R}=R$;
%$\enab_\otimes\left((\loc,\dtloc), (a,r)\right):=\enab(\loc,a)\wedge \phi'$ if $r=\left(\dtloc, b', \phi', Y, \dtloc'\right)$ for some $b', \phi', Y, \dtloc''$, and $\enab_\otimes\left((\loc,\dtloc), (a,r)\right):=\false$ otherwise;
\item 
$
    \lbfunc_\otimes \left(
        \loc,\dtloc
    \right)
        := \left \{
            \dtloc
        \right \}
    \mbox{ for all } \left ( 
        \loc,\dtloc
    \right )
    \in \locs_\otimes
$
\item $\prob_\otimes$ is given by
\begin{align*}
&\prob_\otimes\left((\loc,\dtloc),(a,R)\right)(Y,(\loc',\dtloc')):=\\
&\quad\begin{cases}
\prob\left(\loc,a\right)(Y\cap \clocks_1,\loc') & \mbox{if } (\dtloc,\lbfunc\left(\loc'\right), \phi^{\dtloc,\lbfunc\left(\loc'\right)}_R, Y\cap \clocks_2, \dtloc')\in\rules\\%\mbox{ and }R'={R}{\left[Y\cap \clocks_2:=0\right]}  \\
0 & \mbox{otherwise}
\end{cases}\enskip.
\end{align*}
where $(\dtloc,\lbfunc\left(\loc'\right), \phi^{\dtloc,\lbfunc\left(\loc'\right)}_R, Y\cap \clocks_2, \dtloc')$ is the unique rule such that for all $\nu\in R$, $\nu\in\sat{\phi^{\dtloc,\lbfunc\left(\loc'\right)}_R}$.
The uniqueness follows from determinism and totality of DTAs.
\end{compactitem}


Now we present the finite acceptance of nodeterministic timed automata for PTAs.
\vspace{-0.8em}
\begin{definition}[Finite Acceptance Criterion]
Let $F\subseteq\cstates$ be a set of \emph{final} modes.
An infinite word $w$ is \emph{finitely accepted} by $\dta$ w.r.t the \emph{initial configuration} $(\dtloc,\nu)$ and $F$ if $\run{\dta}{\dtloc,\nu}{w}=\{(\dtloc_n,\nu_n,a_n)\}_{n\in\Nset_0}$ satisfies that $\dtloc_n\in F$ for
some $n\in\Nset_0$.
\end{definition}

\begin{definition}[Path Acceptance]
An infinite path $\infpath$ under $\pta$ is \emph{finitely accepted} by $\tra$ w.r.t 
initial configuration $(\dtloc,\nu)$, if the infinite word $\lbfunc(\infpath)$ is finitely
accepted by $\tra$ w.r.t 
$
\left(\trfunc\left((\dtloc,\nu), \lbfunc(\initloc{\infpath})\right),\zero\right)
$.
\end{definition}


\begin{definition}[floor Operator]
For a region $\reg$ with clocks $\clocks$, $\floor{\reg} : \clocks \rightarrow \Nset$ 
is defined as $ \floor{\reg}(x) = t $ where $t$ is the unique integer s.t. 
$\reg \models t \le x < t+1$
\end{definition}

Below we fix a well-formed PTA $\pta$ taking the form (\ref{eq:pta}) and a NTA $\nta$ taking the form (\ref{eq:nta}) with the difference that the set of clocks for $\pta$ (resp. for $\nta$) is denoted by $\clocksX$ (resp. $\clocksY$).
W.l.o.g., we assume that $\clocksX \cap \clocksY = \emptyset$ and $\alphabet=2^{\ap}$.

Let PTA be $\pta$ with the set $\clocksX$ and the NTA be $\nta$ with the set $\clocksY$.

The transformation to MDP is as follows.

Let $\clocksY$ be a fixed finite set of clocks. We use integer Subscript denote a set of 
new clocks. Formally
$
    \clocksY_k = \left \{
        \left (
            t,y
        \right ) \in \Nset \times \clocksY
        \mid
        t = k
    \right \}
$ for $ k > 0 $.

And $\reg^{\clocksY_k}$ is a region for $\clocksY_k$.

\noindent{\textbf{Product Construction (Infinite-State-MDP ).}}
The \emph{product MDP} $\productmdp{\pta}{\dta_q}$ between $\pta$ and $\dta$ with initial mode $\dtloc$ is defined as the PTA

\newcommand{\clocksN}{
    \clocksX \cup \left(
        \bigcup_{k=1}^{n} \clocksY_k
    \right )
}

The transformation to MDP is follows. A state in $\productmdp{\pta}{\dta_q}$ 
is of the form 
\begin{equation}\label{eq:state0}
    \left (
        \left (
            \loc,\reg^{\clocksX}
        \right )
        ,
        \left (
            (\dtloc_1,\reg_1^{\clocksY_1}),
            \cdots,
            (\dtloc_n,\reg_n^{\clocksY_n})
        \right )
        ,
        \clocksX \cup \left(
            \bigcup_{k=1}^{n} \clocksY_k
        \right )
        ,
        \subseteq
    \right )
\end{equation}

where $n$ is an unbounded natural number, $\loc$ (w.r.t $\reg^{\clocksX}$) is a location (w.r.t a region) in $\pta$, $\dtloc_i$ (w.r.t $\reg_i^{\clocksY_i}$) is a mode (w.r.t a copy of the original region over $\clocksY_i$) in $\nta$, and $\subseteq$ is a linear order over 
$
    \{ 0 \} 
    \cup 
    \clocksX 
    \cup \left(
        \bigcup_{k=1}^{n} \clocksY_k
    \right ) 
$
The intuition is that 
$
\left (
    \loc,\reg^{\clocksX}
\right )
$ 
reflects the region for $\pta$,
$ 
\left (
    (\dtloc_1,\reg_1^{\clocksY_1}),
    \cdots,
    (\dtloc_n,\reg_n^{\clocksY_n})
\right )
$
reflects a power set for $\nta$, and $ \subseteq $ records the relative order between 
clocks in 
$
    \{ 0 \} 
    \cup 
    \clocksX 
    \cup \left(
        \bigcup_{k=1}^{n} \clocksY_k
    \right ) 
$.

For convenience, we use $ \clocksY_0 $ denote $ \clocksX $.

\begin{definition}[Consistency]
A clock valuation $\nu$ is consistent with a state $\mdploc$ in the form of ~(\ref{eq:state0}), 
denoted by $ \nu \in  \mdploc$
iff $\nu \in \clocksN$, $ \nu \downarrow \clocksY_i \in \reg^{\clocksY_i} $ and 
$\fracp{\nu(x)} < \fracp{\nu(y)}$ iff $x \subseteq y$ for all $x,y \in \clocksN$.
\end{definition}

\begin{definition}[Time successor]
A state $\mdploc'$ 
\begin{equation}\label{eq:state1}
    \left (
        \left (
            \loc,(\reg^{\clocksX})'
        \right )
        ,
        \left (
            (\dtloc_1,(\reg_1^{\clocksY_1})'),
            \cdots,
            (\dtloc_n,(\reg_n^{\clocksY_n})')
        \right )
        ,
        \clocksX \cup \left(
            \bigcup_{k=1}^{n} \clocksY_k
        \right )
        ,
        \subseteq'
    \right )
\end{equation} 
is a time successor of $\mdploc$ in the 
form of ~(\ref{eq:state0}) where either 
\begin{compactitem}
    \item 
        $\mdploc = \mdploc'$ if 
        $
            \forall \nu \in \mdploc, t \in \Rset_{>0} : \nu + t \in s
        $ or
    \item 
        $\mdploc'$ is another unique state in MDP if
        there exist a $\nu$ consistent with $\mdploc$ s.t.
        \begin{align*}
            \exists t \in \Rset_{\ge0} : (
                \nu + t \in \mdploc' 
                \land
                \forall t' \in [0,t] : \\ \left (
                    \left (
                        \nu + t' \in \mdploc
                        \lor
                        \nu + t' \in \mdploc'
                    \right )
                    \land
                    \nu + t' \models \inv(l)
                \right )
            )
        \end{align*}
\end{compactitem}
\end{definition}

\begin{definition}[Transition relation]

The \emph{transition relation} $\trans$ is the smallest relation such that the following two inference rules are satisfied : \\
(Delay)
$
    \begin{array}{c}
        \mdploc' \mbox{ is the time successor of } \mdploc \\
        \hline
        \tran
            {\mdploc}
            {\tau}
            {\mu_{\mdploc'}}
    \end{array}
$
\\
(Jump)
$
    \begin{array}{ccc}
        \nu \in \mdploc 
        &
        \tran
            {\left (
                \loc, \nu \downarrow \clocksX
            \right )}
            {a}
            {\mu}
        &
        \nu \downarrow \clocksX \models \penab{\loc}{a}
        \\
        \hline
        &
        \tran
            {\mdploc}
            {a}
            {\mu^*}
        &
    \end{array}
$
\\
where, let 
\\
$$
\mdploc' =  \left (
    \left (
        \loc',(\reg^{\clocksX})'
    \right )
    ,
    \left (
        (\dtloc_{i_0},(\reg_{i_0}^{\clocksY_{i_0}})'),
        \cdots,
        (\dtloc_{i_{k_i}},(\reg_{i_{k_i}}^{\clocksY_{i_{k_i}}})')
    \right )
    ,
    \clocksX \cup \left(
        \bigcup_{i=1}^{n'} \bigcup_{j=0}^{k_i} \clocksY_{i_j}
    \right )
    ,
    \subseteq'
\right )
$$
\\
\begin{align*}
    &
    \mu^* \left (
       \mdploc'
    \right )
    = 
    \begin{cases}
        & 
        \nu(X,\loc')
        \mbox{  following conditions are hold}
        \\
        & 
        0 
        \mbox{  otherwise }
    \end{cases}
\end{align*} 
\\
The none zero case hold
\begin{align*}
    &
    \mbox{ if } (\reg^{\clocksX})' 
                    = 
                    \evclass{ \left (
                            \nu \downarrow \clocksX
                        \right) [X := 0 ]
                    }_\sim 
    \mbox{ , there exists} \left (
        \dtloc_i,
        \lbfunc\left(\loc'\right),
        \phi,
        Y,
        \dtloc_{i_j}
    \right) \in \rules
    \\
    &
    \mbox{ such that }
    (\reg^{\clocksY_{i_j}})' 
        = 
        \evclass{ \left (
                \nu \downarrow \clocksY_{i}
            \right) [Y := 0 ] [y \mapsto <i_j,y>]
        }_\sim 
    \mbox{ and }
    (\reg^{\clocksY_{i_j}})' \subseteq \sat{\phi}
    \mbox{,}
    \\
    &
    \mbox { $ k_i $ is the number of successors of $ \dtloc_i $. } 
    \mbox { and } 
\end{align*} 
% \begin{align*}


% \end{align*}
\end{definition}
