% \vspace{-0.8em}
\section{Conclusion and Future Work}
% \vspace{-0.8em}

In this paper, we studied the problem of model-checking PTAs against timed-automata specifications.
We considered both Rabin and finite acceptance conditions.
For Rabin acceptance condition,  
we first solved the problem with DTA specifications and Rabin acceptance condition through a product construction and prove that its computational
complexity is EXPTIME-complete;
then we proved that the problem with general timed-automata specifications is undecidable through a reduction from the universality problem of timed automata. 
For finite acceptance condition, we demonstrated that the problem with DTA specifications can be solved through efficient zone-based algorithms on verifying reachability probability of PTAs~\cite{DBLP:journals/fmsd/NormanPS13,DBLP:journals/tcs/KwiatkowskaNSS02}, while the problem with general timed-automata specifications can be solved by an approximation algorithm based on value iteration.

An interesting future direction is zone-based algorithms for Rabin acceptance condition.  
Another theoretical direction is to investigate timed-automata specifications with cost or reward.
A more practical direction is to apply our approaches to industrial-level examples. 

\section{Related Works}

Model-checking probabilistic timed models against linear dense-time properties
are mostly considered for continuous-time Markov processes (CTMPs).
First, Donatelli~\emph{et al.}~\cite{DBLP:journals/tse/DonatelliHS09} proved an expressibility result that the class of linear dense-time properties encoded by DTAs is not subsumed by branching-time properties.
They also demonstrated an efficient algorithm for verifying continuous-time Markov chains~\cite{?} against one-clock DTAs.
Then various results on verifying CTMPs are obtained for specifications through DTAs and general timed automata (cf.~\cite{DBLP:journals/tse/DonatelliHS09,DBLP:journals/corr/abs-1101-3694,DBLP:conf/hybrid/Fu13,DBLP:conf/hybrid/BrazdilKKKR11,DBLP:conf/tacas/BarbotCHKM11,DBLP:conf/formats/BortolussiL15}).
The fundamental difference between CTMPs and PTAs is that the former assign probability distributions to time elapses, while the latter treat time-elapses as pure nondeterminism.
Because of this difference, the techniques for CTMPs cannot be applied to PTAs.

For PTAs, the only relevant result is by ?\cite{?} who developed an approach for verifying PTAs against deterministic discrete-time omega-regular automata through a similar product construction.
Our results extend theirs in two ways.
First, our product construction extends theirs with extra ability to tackle timing constraints from both the PTA and the DTA.
The extension is nontrivial since it needs to resolve the integration between randomness and timing constraints, while ensuring the EXPTIME-completeness of the problem, matching the computational complexity in the discrete-time case \cite{?}.
Second, our results also cover an undecidability result and an approximation algorithm in the case of general nondeterministic timed automata, extending \cite{?} with nondeterminism.
