% \vspace{-0.8em}
\section{Conclusion and Future Work}
% \vspace{-0.8em}

In this paper, we studied the problem of model-checking PTAs against timed-automata specifications. 
We first solve the problem with DTA specifications 

through a product construction and prove that the problem is 
EXPTIME-complete. 


linear-time model-checking problem {\sc PTA-DTA} of DTA-specifications over PTAs.
To solve the problem, we gave two versions of product construction (between a PTA and a DTA) which both reduce the problem to computing reachability probabilities over PTAs, for which efficient algorithms exist~\cite{DBLP:journals/fmsd/NormanPS13,DBLP:journals/tcs/KwiatkowskaNSS02}.
Both the product constructions are nontrivial and are elaborated so that one caters for DTAs with a small number of regions and the other for DTAs with small size.
Then we demonstrated two case studies clarifying that the problem PTA-DTA can be applied to real-world applications.
Experimental results show that our product construction is efficient to solve the problem.
A challenging future work is to study the PTA-DTA problem with general timed-automata specifications.
Another future work is to integrate costs or rewards into this problem.

\section{Related Works}

Model-checking probabilistic timed models against linear dense-time properties
are mostly considered for continuous-time Markov processes (CTMPs).
First, Donatelli~\emph{et al.}~\cite{DBLP:journals/tse/DonatelliHS09} proved an expressibility result that the class of linear dense-time properties encoded by DTAs is not subsumed by branching-time properties.
They also demonstrated an efficient algorithm for verifying continuous-time Markov chains~\cite{?} against one-clock DTAs.
Then various results on verifying CTMPs are obtained for specifications through DTAs and general timed automata (cf.~\cite{DBLP:journals/tse/DonatelliHS09,DBLP:journals/corr/abs-1101-3694,DBLP:conf/hybrid/Fu13,DBLP:conf/hybrid/BrazdilKKKR11,DBLP:conf/tacas/BarbotCHKM11,DBLP:conf/formats/BortolussiL15}).
The fundamental difference between CTMPs and PTAs is that the former assign probability distributions to time elapses, while the latter treat time-elapses as pure nondeterminism.
Because of this difference, the techniques for CTMPs cannot be applied to PTAs.

For PTAs, the only relevant result is by ?\cite{?} who developed an approach for verifying PTAs against deterministic discrete-time omega-regular automata through a similar product construction.
Our results extend theirs in two ways.
First, our product construction extends theirs with extra ability to tackle timing constraints from both the PTA and the DTA.
The extension is nontrivial since it needs to resolve the integration between randomness and timing constraints.
Moreover, the extension still leads to the EXPTIME-completeness of the problem, matching the computational complexity in the discrete-time case \cite{?}.
Second, our results also cover an undecidability result and an approximation algorithm in the case of general nondeterministic timed automata, extending \cite{?} with nondeterminism.
