% \vspace{-0.8em}
\section{The Product Construction}
% \vspace{-0.8em}
In this section, we introduce the core part of our algorithms to solve the {\sc PTA-DTA} problem. The core part is a product construction which given a PTA $\pta$ and a DTA $\dta$, output a PTA which preserves the probability of the set of infinite paths of $\pta$ accepted by $\dta$.
Below we fix a well-formed PTA $\pta$ in the form (\ref{eq:pta}) and a DTA $\dta$ in the form (\ref{eq:dta}) 
with the difference that the set of clocks for $\pta$ (resp. for $\dta$) is denoted by $\clocks_1$ (resp. $\clocks_2$).
W.l.o.g., we assume that $\clocks_1\cap\clocks_2=\emptyset$ and $\alphabet=2^{\ap}$.
%such that $\clocks_1\cap\clocks_2=\emptyset$.
%We also fix a reward structure $(\rcum,\rinst)$ for $\pta$.
We let $\regions$ be the set of regions w.r.t $\sim_N$, where $N$ is the maximal integer appearing in the clock constraints of $\dta$.

{\bf The Main Idea.} The intuition of the product construction is to let $\dta$ reads external actions of $\pta$ while $\pta$ evolves along the time axis.
The major difficulty is that when $\pta$ performs actions in $\acts$, there is a probabilistic choice between the target locations. Then $\dta$ needs to know the labelling of the target location and the rule (in $\rules$) used for the transition.
A naive solution is to integrate each single rule $\rules$ into the enabling condition $\enab$ in $\pta$. However, this simple solution does not work since a single rule in $\rules$ fixes the labelling of a location in $\pta$, while the probabilistic distribution given by $\prob$ can jump to locations with different labels.
We solve this difficulty by integrating into the enabling condition $\enab$ enough information on clock valuations under $\dta$ so that the rule  used for the transition (in $\dta$) is clear.
In detail, we introduce two versions of the product construction, each having a computational advantage against the other.

\noindent{\textbf{Product Construction (First Version).}}
The \emph{product PTA} $\product{\pta}{\dta_q}$ between $\pta$ and $\dta$ with initial mode $\dtloc$ is defined as the PTA $\left({\locs}_\otimes, \loc^*_\otimes, \clocks_\otimes, \acts_\otimes, \inv_\otimes, \enab_\otimes,  \prob_\otimes, \lbfunc_\otimes\right)$, where:
\begin{compactitem}
\item $\locs_\otimes:=\locs\times \cstates$;
\item $\loc^*_\otimes:=(\loc^*, q^\star)$ where $q^\star$ is the unique mode such that $\trfunc\left((\dtloc,\zero), \lbfunc(\loc^*)\right)=(q^\star,\zero)$; %and ${\left[\zero\right]}_\sim$ is the region which contains the sole element $\zero$;
\item $\clocks_\otimes:=\clocks_1\cup\clocks_2$;
\item $\acts_\otimes:=\acts\times\regions$; %$\acts_\otimes:=\acts\times\rules$;
\item $\inv_\otimes(\loc,\dtloc):=\inv(\loc)$ for all $(\loc,\dtloc)\in \locs_\otimes$;
\item $\enab_\otimes\left((\loc,\dtloc), (a,R)\right):=\enab(\loc,a)\wedge \phi_R$ for all $(\loc,\dtloc)\in \locs_\otimes$, where $\phi_R$ is any clock constraint such that $\sat{\phi_R}=R$;
%$\enab_\otimes\left((\loc,\dtloc), (a,r)\right):=\enab(\loc,a)\wedge \phi'$ if $r=\left(\dtloc, b', \phi', Y, \dtloc'\right)$ for some $b', \phi', Y, \dtloc''$, and $\enab_\otimes\left((\loc,\dtloc), (a,r)\right):=\false$ otherwise;
\item 
$
    \lbfunc_\otimes \left(
        \loc,\dtloc
    \right)
        := \left \{
            \dtloc
        \right \}
    \mbox{ for all } \left ( 
        \loc,\dtloc
    \right )
    \in \locs_\otimes
$
\item $\prob_\otimes$ is given by
\begin{align*}
&\prob_\otimes\left((\loc,\dtloc),(a,R)\right)(Y,(\loc',\dtloc')):=\\
&\quad\begin{cases}
\prob\left(\loc,a\right)(Y\cap \clocks_1,\loc') & \mbox{if } (\dtloc,\lbfunc\left(\loc'\right), \phi^{\dtloc,\lbfunc\left(\loc'\right)}_R, Y\cap \clocks_2, \dtloc')\in\rules\\%\mbox{ and }R'={R}{\left[Y\cap \clocks_2:=0\right]}  \\
0 & \mbox{otherwise}
\end{cases}\enskip.
\end{align*}
where $(\dtloc,\lbfunc\left(\loc'\right), \phi^{\dtloc,\lbfunc\left(\loc'\right)}_R, Y\cap \clocks_2, \dtloc')$ is the unique rule such that for all $\nu\in R$, $\nu\in\sat{\phi^{\dtloc,\lbfunc\left(\loc'\right)}_R}$.
The uniqueness follows from determinism and totality of DTAs.
\end{compactitem}

Apart from standard constructions (e.g., the Cartesian product between $\locs$ and $\cstates$), the product construction also has Cartesian product between $\acts$ and $\regions$. Then for each extended action $(a,R)$, the enabling condition for this action is just the conjunction between $\enab(\loc,a)$ and $R$.
This is to ensure that when the action $(a,R)$ is taken, the clock valuation under $\dta$ lies in $R$.
Finally in the definition for $\prob_\otimes$, upon the action $(a,R)$ and the target location $\loc'$, the DTA $\dta$ chooses the unique rule $(\dtloc,\lbfunc\left(\loc'\right), \phi^{\dtloc,\lbfunc\left(\loc'\right)}_R, Y\cap \clocks_2, \dtloc')$ and then jump to $\dtloc'$ with reset set $Y\cap \clocks_2$.
By integrating regions into the enabling condition, the DTA $\dta$ can know the status of the clock valuation under $\dta$ through its region, hence can decide which rule to use for the transition.
This version of product construction works well if the number of regions is not large.
We note that the number of regions only depends on $N$, not on the size of $\dta$.
In the following, we introduce another version which depends directly on the size of $\dta$.
The second version has an advantage when the number of regions is large.

\noindent{\textbf{Product Construction (Second Version).}}
For each $\dtloc\in\cstates$, we let
\[
\exttuples_{\dtloc}:=\{h:\alphabet\rightarrow\clcons{\clocks_2}\mid\forall b\in\alphabet.\left(\dtloc, b, h(b), X, \dtloc')\in\rules\mbox{ for some }X, \dtloc'\right)\}\enskip.
\]
Intuitively, every element of $\exttuples_{\dtloc}$ is a tuple of clock constraints $\{\phi_b\}_{b\in\alphabet}$, where each
clock constraint $\phi_b$ is chosen from the rules emitting from $\dtloc$ and $b$.
%Let
%\[
%\extactions_\dtloc:= \left\{\phi\in \clcons{\clocks_2}\mid \exists(\phi_1,\dots,\phi_{|\alphabet|})\in\extactions_{\dtloc}.\phi=\bigwedge_{k=1}^{|\alphabet|} \phi_k\mbox{ and }\phi\mbox{ is satisfiable}\right\}\enskip.
%\]
%Thus, $\extactions_\dtloc$ consists of conjunctions of tuples from $\bigcup_{\dtloc\in\cstates}\exttuples_\dtloc$.
%The motivation to introduce $\extactions_\dtloc$ is that once $\nu\models\phi\in $
%Intuitively, clock constraints from $\extactions$ explicitly identifies the reset set and the target mode of a rule once
The \emph{product PTA} $\product{\pta}{\dta_q}$ between $\pta$ and $\dta$ with initial mode $\dtloc$ is defined
almost the same as in the first version of the product construction, with the following differences:
\begin{compactitem}
\item $\acts_\otimes:=\acts\times\bigcup_\dtloc\exttuples_\dtloc$; %$\acts_\otimes:=\acts\times\rules$;
\item $\enab_\otimes\left((\loc,\dtloc), (a,h)\right):=\enab(\loc,a)\wedge \bigwedge_{b\in\alphabet} h(b)$ for all $(\loc,\dtloc)\in \locs_\otimes$ and $h\in \exttuples_\dtloc$, and $\enab_\otimes\left((\loc,\dtloc), (a,h)\right):=\false$ otherwise;
%$\enab_\otimes\left((\loc,\dtloc), (a,r)\right):=\enab(\loc,a)\wedge \phi'$ if $r=\left(\dtloc, b', \phi', Y, \dtloc'\right)$ for some $b', \phi', Y, \dtloc''$, and $\enab_\otimes\left((\loc,\dtloc), (a,r)\right):=\false$ otherwise;
\item $\prob_\otimes$ is given by
\begin{align*}
&\prob_\otimes\left((\loc,\dtloc),(a,h)\right)(Y,(\loc',\dtloc')):=\\
&\quad\begin{cases}
\prob\left(\loc,a\right)(Y\cap \clocks_1,\loc') & \mbox{if } (\dtloc,\lbfunc\left(\loc'\right), h(\lbfunc\left(\loc'\right)), Y\cap \clocks_2, \dtloc')\in\rules\\%\mbox{ and }R'={R}{\left[Y\cap \clocks_2:=0\right]}  \\
0 & \mbox{otherwise}
\end{cases}\enskip.
\end{align*}
\end{compactitem}
The intuition for the second version is that it is also possible to specify the information needed to identify the rule to be chosen by the DTA through a local conjunction of the rules emitting from a mode.
For each mode, the local conjunction chooses one clock constraint from rules with the same symbol, and group them together through conjunction.
From determinism and totality of DTAs,
each conjunction constructed in this way determines which rule to use in the DTA for every symbol in a unique way.
The advantage of the second version against the first one is that it is more suitable for DTAs with small size and large $N$ (leading to a large number of reigons), as the size of the product PTA relies only the size of the DTA.
\vspace{-0.8em}
\begin{remark}
It is easy to see that the PTA $\product{\pta}{\dta_q}$ (in both versions) is well-formed as $\pta$ is well-formed and the DTA $\dta$ does not introduce extra invariant conditions.
\end{remark}
\vspace{-0.8em}



%$\prob_\otimes$ satisfies that for any $\loc,\loc''\in\locs$, $\dtloc,\dtloc''\in\cstates$, $a\in\acts$, $r=\left(\dtloc, \lbfunc(\loc''), \phi', Y, \dtloc'\right)\in\rules$, $(X,\loc')\in 2^{\clocks_1}\times \locs$ and $Z\subseteq\clocks$, it holds that
%\[
%\prob_\otimes\left((\loc,\dtloc),(a,r)\right)(Z,(\loc'',\dtloc''))=\begin{cases}
%\prob\left(\loc,a\right)({Z}{\setminus}{Y},\loc'') & \mbox{if } \dtloc''=\dtloc' \\
%0 & \mbox{otherwise}
%\end{cases}\enskip.
%\]

In the following, we clarify the relationship between $\pta,\dta$ and $\product{\pta}{\dta_\dtloc}$.
We first show the relationship between paths under $\pta$ and paths under $\product{\pta}{\dta_\dtloc}$.
Informally, paths under $\product{\pta}{\dta_\dtloc}$ are just paths under $\pta$ extended with runs of $\dta$.

{\textbf{Transformation $\pfunc$ From Paths under $\pta$ into Paths under $\product{\pta}{\dta_\dtloc}$.}}
Since the two versions of product construction shares similarities, we illustrate the transformation in a unified fashion.
The transformation is defined as the function $\pfunc:\fnpaths{\pta}\cup\infpaths{\pta}\rightarrow \fnpaths{\product{\pta}{\dta_\dtloc}}\cup\infpaths{\product{\pta}{\dta_\dtloc}}$
which transform a finite or infinite path under $\pta$ into one under  $\product{\pta}{\dta_\dtloc}$ as follows.
For a finite path
\[
\fnpath=(\loc_0,\nu_0)a_0\dots a_{n-1}(\loc_n,\nu_n)
\]
under $\pta$ (note that $(\loc_0,\nu_0)=(\loc^*, \zero)$ by definition),
we define $\pfunc(\fnpath)$ to be the unique finite path
\begin{equation}\label{eq:trho}
\pfunc(\fnpath):=((\loc_0,\dtloc_0),\nu_0\cup\mu_0)a'_0\dots a'_{n-1}((\loc_n,\dtloc_n),\nu_n\cup\mu_n)
\end{equation}
under $\product{\pta}{\dta_\dtloc}$ such that (\dag)
\begin{compactitem}
\item $\trfunc\left((\dtloc,\zero), \lbfunc(\loc^*)\right)=(q_0,\mu_0)$ (note that $\mu_0=\zero$), and
\item for all $0\le k< n$, if $a_k\in [0,\infty)$ then $a'_k=a_k$ and $\dtatr{(\dtloc_k,\mu_k)}{a_k}{\left(\dtloc_{k+1},\mu_{k+1}\right)}$, and
\item for all $0\le k< n$, if $a_k\in\acts$ then $a'_k=(a_k,\xi_k)$ and $\dtatr{(\dtloc_k,\mu_k)}{\lbfunc(\loc_{k+1})}{\left(\dtloc_{k+1},\mu_{k+1}\right)}$, where either (i) the first version of the product construction is taken and $\xi_k$ is the region $\left[\mu_k\right]_\sim$ or (ii) the second version is taken and $\xi_k$ is the unique function such that for each symbol $b\in\alphabet$, $\xi_k(b)$ is the unique clock constraint appearing in a rule emitting from $q_k$ and with symbol $b$ such that $\mu_k\models\xi_k(b)$.
\end{compactitem}
Likewise, for an infinite path $\infpath=(\loc_0,\nu_0)a_0(\loc_1,\nu_1)a_1\dots$
under $\pta$, we define $\pfunc(\infpath)$ to be the unique infinite path
\[
\pfunc(\infpath):=((\loc_0,\dtloc_0),\nu_0\cup\mu_0)a'_0((\loc_1,\dtloc_1),\nu_1\cup\mu_1)a'_1\dots
\]
under $\product{\pta}{\dta_\dtloc}$ such that the three conditions below (\dag) hold for all $k\in\Nset_0$ instead of all $0\le k< n$.
\qed

The following lemma shows that $\pfunc$ is a bijection and preserves zenoness.

\begin{lemma}\label{lemm:pfuncbij}
The function $\pfunc$ is a bijection. Moreover, for any infinite path $\infpath$, $\infpath$ is non-zeno iff $\pfunc(\infpath)$ is non-zeno.
\end{lemma}
\begin{proof}
The first claim follows straightforwardly from the determinism and totality of DTAs.
The second claim follows from the fact that $\pfunc$ preserves time elapses in the transformation.
\qed
\end{proof}

%between $\fnpaths{\pta,\sigma}\cup\infpaths{\pta,\sigma}$ and $\fnpaths{\product{\pta}{\dta_q},\sfunc\left(\sigma\right)}\cup\infpaths{\product{\pta}{\dta_\dtloc},\sfunc\left(\sigma\right)}$.

We also show the relationship on schedulers before and after product construction.

\noindent{\textbf{Transformation $\sfunc$ From Schedulers under $\pta$ into Schedulers under $\product{\pta}{\dta_\dtloc}$.}}
We define the function $\sfunc$ from the set of schedulers under $\pta$ into the set of schedulers under $\product{\pta}{\dta_\dtloc}$ as follows: for any scheduler $\sigma$ for $\pta$, $\sfunc(\sigma)$ (for $\product{\pta}{\dta_\dtloc}$) is defined such that for any finite path $\fnpath$ under $\pta$ where $\fnpath=(\loc_0,\nu_0)a_0\dots a_{n-1}(\loc_n,\nu_n)$ and $\pfunc(\rho)$ is given as in (\ref{eq:trho}),
\[
\sfunc(\sigma)(\pfunc(\fnpath)):=
\begin{cases}
\sigma(\fnpath) & \mbox{if }n\mbox{ is even} \\
(\sigma(\fnpath),\lambda(\rho)) & \mbox{if }n\mbox{ is odd}
\end{cases}
\]
where $\lambda(\rho)$ is either $\left[\mu_n\right]_\sim$ if the first version of the product construction is taken, or
the unique function such that for each symbol $b\in\alphabet$, $\lambda(\rho)(b)$ is the unique clock constraint appearing in a rule emitting from $q_k$ and with symbol $b$ such that $\mu_n\models\lambda(\rho)(b)$.
Note that the well-definedness of $\sfunc$ follows from Lemma~\ref{lemm:pfuncbij}.\qed

By Lemma~\ref{lemm:pfuncbij}, the product construction and the determinism and totality of DTAs, one can prove straightforwardly the following lemma.
\vspace{-0.8em}
\begin{lemma}\label{lemm:sfuncbij}
The function $\sfunc$ is a bijection.
\end{lemma}
\vspace{-0.8em}
Now we show the relationship between infinite paths accepted by a DTA before product construction and infinite paths visiting certain target locations after product construction.
Below we lift the function $\pfunc$ to all subsets of paths in the standard fashion: for all subsets $A\subseteq \fnpaths{\pta}\cup\infpaths{\pta}$, $\pfunc(A):=\{\pfunc(\omega)\mid \omega\in A\}$.

%%%%% 
\vspace{-0.8em}
\begin{definition}[Traces]
Let 
$ \pfunc(\infpath) = 
    ((\loc_0,\dtloc_0),\nu_0\cup\mu_0)
    a'_0
    ((\loc_1,\dtloc_1),\nu_1\cup\mu_1)
    a'_1
    \dots 
$
% and according to definition 
% $Lab((\loc_i,\dtloc_i)) = \left\{ \dtloc_i \right\}$.
% Instead of using $\{ q_0 \}, \{ q_1 \}, \dots $, 
the trace of $ \pfunc ( \infpath ) $ is defined by
$\trace{ \pfunc ( \infpath ) } := \dtloc_0 \dtloc_1 \dots $.
\end{definition}
%%%%%

% \vspace{-0.8em}
% \begin{proposition}\label{prop:psfunc} $\pfunc\left(\acc{\pta,\sigma}{\dta,\dtloc,F}\right)=\omgpaths{\product{\pta}{\dta_\dtloc},\sfunc\left(\sigma\right)}{\locs\times F}$ for any scheduler $\sigma$, initial mode $q$ and $F\subseteq \cstates$. %or in Definition~\ref{def:infacc}.
% \end{proposition}
% \begin{proof}
% Directly from the product construction and Definition~\ref{def:fnacc}.\qed
% \end{proof}

\noindent{\textbf{Verifying Limit Rabin Properties.}}
Paths in $\product{\pta}{\dta_\dtloc}$ that $\pta$ is accepted by $\dta$ with $\rabin$ is
% 写成定义在普通的 PTA 上还是乘积上好
$$
    \AcceptCxAqsF = \left \{ 
        \infpath \in \infpaths{\product{\pta}{\dta_\dtloc},\sigma} \mid 
        \infset{ 
            \trace{ 
                \infpath
            }
        }
        \mbox{ is Rabin accepting by } \rabin
    \right\}
$$
and $\AcceptCxAqsF$ is an limit LT Property.

\vspace{-0.8em}
\begin{proposition}\label{prop:psfunc} 
For any scheduler $\sigma$, any initial mode $q$ and any Rabin acceptance condition
$\rabin$ on DTA $\dta$, $\TLang = \TAcc$ .
\end{proposition}

\begin{proof}
% This proof is trivial.
By definition we have
$$
    \LangCsAqF = \left \{ 
        \infpath \in \infpaths{\pta,\sigma} \mid 
        \infset{ 
            \traj{ 
                \run{\dta}{\iconfig}{\lbfunc(\infpath)} 
            }
        }
        \mbox{ is Rabin accepting by } \rabin 
    \right\},
$$
where
$
    \dtloc^*
        =
            \trfunc \left(
                (\dtloc,\zero), 
                \lbfunc (
                    \initloc{\infpath}
            \right)
$.
Let $\infpath = (\loc_0,\nu_0) a_0 (\loc_1,\nu_1) a_1 \dots $ be any infinite path.
And by definition of $\pfunc$ we have
$$
    \pfunc( \infpath ) = 
        ((\loc_0,\dtloc_0),\nu_0\cup\mu_0)
        a'_0
        ((\loc_1,\dtloc_1),\nu_1\cup\mu_1)
        a'_1
        \dots 
$$
% 我觉得不需要把验证 T(pi) 跟 theta(sigma) 相容 写下来
$$
    \run{\dta}{\iconfig}{\lbfunc(\infpath)} 
        = \{(\dtloc_n,\mu_n,\lbfunc(\infpath)_{n})\}_{n\in\Nset_0}.
$$
Then it's obvious that
$$\trace{ \pfunc( \infpath ) }
    = q_0 q_1 \dots
    = \traj{ 
        \run
            {\dta}{\iconfig}
            {\lbfunc(\infpath)} 
    }.
$$ 
Then we conclude that
$\infset {
    \trace {
        \pfunc( \infpath )
    }
}$
is Rabin accepting by $\rabin$ iff 
$
    \infset{ 
        \traj { 
            \run
                {\dta}{\iconfig}
                {\lbfunc(\infpath)} 
        }
    }
$
is Rabin accepting by $\rabin$.
\end{proof}

Finally, we demonstrate the relationship between acceptance probabilities before product construction and reachability probabilities after product construction.
We also clarify the probability of zenoness before and after the product construction.

\vspace{-0.8em}
\begin{theorem}\label{thm:main}
For any scheduler $\sigma$, initial mode $q$ and Rabin acceptance $\rabin$,
\[
    \pr
        {\dtloc,\rabin}
        {\sigma}
        =
            \probm
                ^{\pta,\sigma}
                \left(
                % \acc{\pta,\sigma}{\dta,q,F}
                    \LangCsAqF
                \right)
        =
            \probm
                ^{\product{\pta}{\dta_\dtloc},\theta(\sigma)}
                \left(
                    % \omgpaths{\product{\pta}{\dta_\dtloc},\theta\left(\sigma\right)}{\locs\times F}
                    \TAcc
                \right) 
    ~. 
\]
Moreover, 
$
    \probm
        ^{\pta,\sigma}
        \left( \{
                \infpath \mid \infpath \mbox{ is zeno}
            \}
        \right)
    =
    \probm
        ^{\product{\pta}{\dta_\dtloc},\theta(\sigma)}
        \left( \{  
                \infpath' \mid \infpath' \mbox{ is zeno}
            \}
        \right)
$
\enskip.
\end{theorem}
\vspace{-0.8em}
\begin{proof}
Define the probability measure $\probm'$ by: $\probm'(A)=\probm^{\product{\pta}{\dta_\dtloc},\sfunc\left(\sigma\right)}(\pfunc(A))$ for $A\in\mathcal{F}^{\pta,\sigma}$. We show that $\probm'=\probm^{\pta,\sigma}$. By \cite[Theorem 3.3]{PBMeasure}, it suffices to consider cylinder sets as they form a pi-system (cf. \cite[Page 43]{PBMeasure}).
Let $\fnpath=(\loc_0,\nu_0)a_0\dots a_{n-1}(\loc_n,\nu_n)$ be any finite path under $\pta$.
By definition, we have that
\[
\probm^{\pta,\sigma}(\cyl(\fnpath))=\probm^{\product{\pta}{\dta_\dtloc}, \sfunc\left(\sigma\right)}(\cyl(\pfunc(\fnpath)))=\probm^{\product{\pta}{\dta_\dtloc},\sfunc\left(\sigma\right)}(\pfunc(\cyl(\fnpath)))=\probm'(\cyl(\fnpath))\enskip.
\]
The first equality comes from the fact that both versions of product construction preserves transition probabilities. The second equality is due to $\cyl(\pfunc(\fnpath))= \pfunc(\cyl(\fnpath))$.
The final equality follows from the definition.
Hence $\probm^{\pta,\sigma}=\probm'$.
Then the first claim follows from Proposition~\ref{prop:psfunc} and the second claim follows from Lemma~\ref{lemm:pfuncbij}. \qed
\end{proof}
Note that a side result from Theorem~\ref{thm:main} says that $\sfunc$ preserves time-divergence for schedulers before and after product construction.
From Theorem~\ref{thm:main} and Lemma~\ref{lemm:sfuncbij}, one immediately obtains the following result which transforms the {\sc PTA-DTA} problem into computing reachability probabilities under the product PTA.

\vspace{-0.8em}
\begin{corollary}{(\cite{DBLP:conf/qest/Sproston11})}\label{crly:opt}
For any initial mode $q$ and any Rabin acceptance condition $\rabin$,
there exists an $\rabin_* \subseteq \locs_\otimes$ s.t. $\rabin_*$ is a union of several maximal end
components that satisfy $\rabin$. 
\begin{align*}
    \opt_\sigma\pr{\dtloc,\rabin}{\sigma}
        & =  
            \opt_{\sigma'}
            \probm^{\product{\pta}{\dta_\dtloc},\sigma'}\left(
                \Accept
                    {\product{\pta}{\dta_\dtloc},\sigma'}
                    {\rabin}
            \right) \\
        & = 
            \opt_{\sigma''}
            \probm^{\product{\pta}{\dta_\dtloc},\sigma''}\left(
                \omgpaths
                    {\product{\pta}{\dta_\dtloc},\sigma''}
                    {\rabin_*} 
            \right) \\
        & = 
            \opt_{\sigma'''}
            \probm^{\product{\pta}{\dta_\dtloc},\sigma'''}\left(
                \omgpaths
                    {\product{\pta}{\dta_\dtloc},\sigma'''}
                    {\rabin_*}
            \right),
\end{align*}
where $\opt$ refers to either $\inf$ (infimum) or $\sup$ (supremum),
$\sigma$ (resp. $\sigma'$) range over all time-divergent schedulers 
for $\pta$ (resp. $\product{\pta}{\dta_\dtloc}$) and $\sigma''$ (resp. 
$\sigma'''$) range over all time-divergent (resp. all time-divergent and 
time-convergent) schedulers for $\product{\pta}{\dta_\dtloc}$. $\rabin_*$ can be 
resolved by an MEC algorithm and $\opt_\sigma\pr{\dtloc,\rabin}{\sigma}$
can be calculated by a reachability analysis.
\end{corollary}
The way \cite{DBLP:conf/qest/Sproston11} discards time-convergent path
is making a copy of every location in PTA model and enforcing a transition 
from the original one to the copy happen when 1 time unit is passed. After 
transiting to the copy, A transition back to the original one will immediately
happend with no delay. And we put a label \textit{tick} in copy. We only deal with 
paths that satisfy $ \square \Diamond tick $ (i.e. \textit{tick} is satisfied 
infinitely many times).

Using a standard MEC algorithm, we can find all MECs satisfy the corresponding
property of an Rabin acceptance condition. In order to guarantee time-divergence,
we only pick up MECs with at least one location that has an \textit{tick} label and 
let $\rabin_*$ be the union of those MECs.
%
%\vspace{-0.8em}
%
%The reward structure $(\rcum',\rinst')$ for $\pta$ is defined as follows:
%\begin{compactitem}
%\item $\rcum'(\loc,\dtloc):=\rcum(\loc)$ for any $(\loc,\dtloc)\in \locs_\otimes$;
%\item $\rinst'((\loc,\dtloc),(a,h)):=\rinst(\loc,a)$ for any $(\loc,\dtloc)\in \locs_\otimes$ and $(a,h)\in \acts_\otimes$.
%\end{compactitem}




