\vspace{-1.6em}
\section{The PTA-DTRA Problem}
\vspace{-1em}
In this section, we solve the {\sc PTA-DTRA} problem through a product construction.
Based on the product construction, we also settle the complexity of the problem.
Below we fix a well-formed PTA $\pta$ in the form (\ref{eq:pta}) and a DTRA $\dta$ in the form (\ref{eq:tra}).
W.l.o.g, we consider that $\clocks\cap\dtclocks=\emptyset$ and $\alphabet=2^{\ap}$.
%such that $\clocks_1\cap\clocks_2=\emptyset$.
%We also fix a reward structure $(\rcum,\rinst)$ for $\pta$.
%We let $\regions$ be the set of regions w.r.t $\sim_N$, where $N$ is the maximal integer appearing in the clock constraints of $\dta$.

\smallskip
\noindent {\bf The Main Idea.} The core part of the product construction is a PTA which preserves the probability of the set of infinite paths accepted by $\dta$.
The intuition is to let $\dta$ reads external actions of $\pta$ while $\pta$ evolves along the time axis.
The major difficulty is that when $\pta$ performs actions in $\acts$, there is a probabilistic choice between the target locations. Then $\dta$ needs to know the labelling of the target location and the rule (in $\rules$) used for the transition.
A naive solution is to integrate each single rule in $\rules$ into the enabling condition $\enab$ in $\pta$. However, this simple solution does not work since a single rule fixes the labelling of a location in $\pta$, while the probability distribution (given by $\prob$) can jump to locations with different labels.
We solve this difficulty by integrating into the enabling condition
%$\enab$
enough information on clock valuations under $\dta$ so that the rule used for the transition is clear.
%In detail, we introduce two versions of the product construction, each having a computational advantage against the other.

\smallskip
\noindent{\textbf{The Product Construction.}}
For each $\dtloc\in\cstates$, we let
\begin{align*}
\exttuples_{\dtloc}:=\{h:\alphabet\rightarrow\clcons{\dtclocks}\mid
\forall b\in\alphabet.\left(\dtloc, b, h(b), X, \dtloc')\in\rules\mbox{ for some }X, \dtloc'\right)\}\enskip.
\end{align*}
The totality of $\rules$ ensures that $\exttuples_{\dtloc}$ is non-empty.
Intuitively, every element of $\exttuples_{\dtloc}$ is a tuple of clock constraints $\{\phi_b\}_{b\in\alphabet}$, where each
clock constraint $\phi_b$ is chosen from the rules emitting from $\dtloc$ and $b$.
The \emph{product PTA} $\product{\pta}{\dta_q}$ (between $\pta$ and $\dta$ with initial mode $\dtloc$) is defined as
$
\left({\locs}_\otimes, \loc^*_\otimes, \clocks_\otimes, \acts_\otimes, \inv_\otimes, \enab_\otimes,  \prob_\otimes, \cstates, \lbfunc_\otimes\right)
$
where :
% $\locs_\otimes:=\locs\times \cstates$,
% $\loc^*_\otimes:=(\loc^*, q^\star)$ where $q^\star$ is the unique mode such that $\dtatr{(\dtloc,\zero)}{\lbfunc(\loc^*)}{(q^\star,\zero)}$,
%     %and ${\left[\zero\right]}_\sim$ is the region which contains the sole element $\zero$;
% $\clocks_\otimes:=\clocks\cup\dtclocks$,
% $\acts_\otimes:=\acts\times\bigcup_\dtloc\exttuples_\dtloc$, %$\acts_\otimes:=\acts\times\rules$;
% $\inv_\otimes(\loc,\dtloc):=\inv(\loc)$ for all $(\loc,\dtloc)\in \locs_\otimes$,
% $\enab_\otimes\left((\loc,\dtloc), (a,h)\right):=\enab(\loc,a)\wedge \bigwedge_{b\in\alphabet} h(b)$ for all $(\loc,\dtloc)\in \locs_\otimes$ and $h\in \exttuples_\dtloc$, and $\enab_\otimes\left((\loc,\dtloc), (a,h)\right):=\false$ otherwise,
% %$\enab_\otimes\left((\loc,\dtloc), (a,r)\right):=\enab(\loc,a)\wedge \phi'$ if $r=\left(\dtloc, b', \phi', Y, \dtloc'\right)$ for some $b', \phi', Y, \dtloc''$, and $\enab_\otimes\left((\loc,\dtloc), (a,r)\right):=\false$ otherwise;
% $
%     \lbfunc_\otimes \left(
%         \loc,\dtloc
%     \right)
%         := \left \{
%             \dtloc
%         \right \}
%     \mbox{ for all } \left (
%         \loc,\dtloc
%     \right )
%     \in \locs_\otimes
% $ 
% and 
% $\prob_\otimes$ is given by
% \begin{align*}
% &\prob_\otimes\left((\loc,\dtloc),(a,h)\right)(Y,(\loc',\dtloc')):=\\
% &~~\begin{cases}
% \prob\left(\loc,a\right)(Y\cap \clocks,\loc') & \mbox{if } (\dtloc,\lbfunc\left(\loc'\right), h(\lbfunc\left(\loc'\right)), Y\cap \dtclocks, \dtloc')\\
% & \mbox{\quad is a (unique) rule in }\rules\\%\mbox{ and }R'={R}{\left[Y\cap \clocks_2:=0\right]}  \\
% 0 & \mbox{otherwise}
% \end{cases}\enskip.
% \end{align*}

\begin{compactitem}
\item $\locs_\otimes:=\locs\times \cstates$;
\item $\loc^*_\otimes:=(\loc^*, q^\star)$ where $q^\star$ is the unique mode such that $\dtatr{(\dtloc,\zero)}{\lbfunc(\loc^*)}{(q^\star,\zero)}$;
    %and ${\left[\zero\right]}_\sim$ is the region which contains the sole element $\zero$;
\item $\clocks_\otimes:=\clocks\cup\dtclocks$;
\item $\acts_\otimes:=\acts\times\bigcup_\dtloc\exttuples_\dtloc$; %$\acts_\otimes:=\acts\times\rules$;
\item $\inv_\otimes(\loc,\dtloc):=\inv(\loc)$ for all $(\loc,\dtloc)\in \locs_\otimes$;
\item $\enab_\otimes\left((\loc,\dtloc), (a,h)\right):=\enab(\loc,a)\wedge \bigwedge_{b\in\alphabet} h(b)$ if $h\in \exttuples_\dtloc$,and $\enab_\otimes\left((\loc,\dtloc), (a,h)\right):=\false$ otherwise, for all $(\loc,\dtloc)\in \locs_\otimes$, $(a,h) \in  \acts_\otimes$.
%$\enab_\otimes\left((\loc,\dtloc), (a,r)\right):=\enab(\loc,a)\wedge \phi'$ if $r=\left(\dtloc, b', \phi', Y, \dtloc'\right)$ for some $b', \phi', Y, \dtloc''$, and $\enab_\otimes\left((\loc,\dtloc), (a,r)\right):=\false$ otherwise;
\item
$
    \lbfunc_\otimes \left(
        \loc,\dtloc
    \right)
        := \left \{
            \dtloc
        \right \}
    \mbox{ for all } \left (
        \loc,\dtloc
    \right )
    \in \locs_\otimes;
$
\item $\prob_\otimes$ is given by
\begin{align*}
&\prob_\otimes\left((\loc,\dtloc),(a,h)\right)(Y,(\loc',\dtloc')):=\\
&~~\begin{cases}
\prob\left(\loc,a\right)(Y\cap \clocks,\loc') & \mbox{if } (\dtloc,\lbfunc\left(\loc'\right), h(\lbfunc\left(\loc'\right)), Y\cap \dtclocks, \dtloc')\\
& \mbox{\quad is a (unique) rule in }\rules\\%\mbox{ and }R'={R}{\left[Y\cap \clocks_2:=0\right]}  \\
0 & \mbox{otherwise}
\end{cases}\enskip.
\end{align*}
for all $(\loc,\dtloc),(\loc',\dtloc')\in \locs_\otimes$, $(a,h) \in  \acts_\otimes$ , $Y \in \clocks_\otimes$
\end{compactitem}
Besides standard constructions (e.g., the Cartesian product between $\locs$ and $\cstates$), the product construction also has Cartesian product between $\acts$ and $\bigcup_\dtloc\exttuples_\dtloc$. For each extended action $(a,h)$, the enabling condition for this action is the conjunction between $\enab(\loc,a)$ and all clock constraints from $h$.
This is to ensure that when the action $(a,h)$ is taken, the clock valuation under $\dta$ satisfies every clock constraint in $h$.
Then in the definition for $\prob_\otimes$, upon the action $(a,h)$, the product PTA first perform probabilistic jump from $\pta$ with the target location $\loc'$, then chooses the unique rule $(\dtloc,\lbfunc\left(\loc'\right), h(\lbfunc\left(\loc'\right)), Y\cap \dtclocks, \dtloc')$ from the emitting mode $\dtloc$ and the label $\lbfunc\left(\loc'\right)$ for which the uniqueness comes from the determinism of $\rules$, then perform the discrete transition from $\dta$.
Finally, we label each $(\loc,\dtloc)$ by $\dtloc$ to meet the Rabin acceptance condition.\qed

It is easy to see that the PTA $\product{\pta}{\dta_q}$ is well-formed as $\pta$ is well-formed and $\dta$ does not introduce extra invariant conditions.
\begin{figure*}
    \centering
    \resizebox{1.0\textwidth}{!}{
        \def \aY {8}
\def \hY {7}
\def \bY {6}
\def \mY {4}
\def \cY {2}
\def \lY {1}
\def \dY {0}

\def \workX {0}
\def \decX  {4}
\def \doneX {6}

\begin{tikzpicture}[x = 1.2cm]

\location{\PairV{a1}{q}}{(\workX,\mY)}
    {
        \PairS
            {\WORK{\alpha}}
            {\Q}
    }
    { \true }
    { \emptyset };

\node[dec] (adec) at (\decX,\aY) {$\bullet$};
\node[dec] (bdec) at (\decX,\bY) {$\bullet$};
\node[dec] (cdec) at (\decX,\cY) {$\bullet$};
\node[dec] (ddec) at (\decX,\dY) {$\bullet$};


\location{\PairV{a2}{q}}{(\doneX,\hY)}
    {
        \PairS
            {\DONE{\alpha}}
            {\Q}
    }
    { x=0 }
    { \{ \Q \} };

\location{\PairV{a2}{f}}{(\doneX,\lY)}
    {
        \PairS
            {\DONE{\alpha}}
            {\FAIL}
    }
    { x=0 }
    { \{ \FAIL \} };

%%%%% h0
\draw[nchoice] (\PairV{a1}{q}) -- (\workX,\aY) to node[auto,sloped,above] {
    $\PairS{\tau}{h_0},( y \le 500 ) \land ( z \le 300 ) $
} (adec);

\draw[pchoice] (adec) to node[auto,sloped,above] {
    $ \{ x \},\p{\alpha}$
} (\PairV{a2}{q});

\draw[pchoice,bend right] (adec) to node[auto,above,sloped] {
    $ \emptyset, 1 - \p{\alpha}$
} (\PairV{a1}{q});

%%%%% h1
\draw[nchoice] (\PairV{a1}{q}) to node[auto,sloped,above] {
    $\PairS{\tau}{h_1},( y \le 500 ) \land ( 300 < z ) $
} (bdec);

\draw[pchoice] (bdec) to node[auto,sloped,above] {
    $ \{ x \},\p{\alpha}$
} (\PairV{a2}{q});

\draw[pchoice,bend left] (bdec) to node[auto,above,sloped] {
    $ \emptyset, 1 - \p{\alpha}$
} (\PairV{a1}{q});

%%%%% h2
\draw[nchoice] (\PairV{a1}{q}) to node[auto,sloped,above] {
    $\PairS{\tau}{h_2},( 500 < y ) \land ( z \le 300 ) $
} (cdec);

\draw[pchoice] (cdec) to node[auto,sloped,above] {
    $ \{ x \},\p{\alpha}$
} (\PairV{a2}{f});

\draw[pchoice,bend left] (cdec) to node[auto,above,sloped] {
    $ \emptyset, 1 - \p{\alpha}$
} (\PairV{a1}{q});

%%%%% h3
\draw[nchoice] (\PairV{a1}{q}) -- (\workX,\dY) to node[auto,sloped,above] {
    $\PairS{\tau}{h_2},( 500 < y ) \land ( z \le 300 ) $
} (ddec);

\draw[pchoice] (ddec) to node[auto,sloped,above] {
    $ \{ x \},\p{\alpha}$
} (\PairV{a2}{f});

\draw[pchoice,bend left] (ddec) to node[auto,above,sloped] {
    $ \emptyset, 1 - \p{\alpha}$
} (\PairV{a1}{q});

\end{tikzpicture}
        }
    \caption{The Product PTA for Our Running Example}
    \label{fig:product}
\end{figure*}
\begin{example}
The product PTA between the PTA in Example~\ref{ex:pta} and the DTRA in Example~\ref{ex:dta} is depicted in Figure~\ref{fig:product}.
In the figure, $\PairS{\WORK{\alpha}}{\q{\alpha}}, \PairS{\WORK{\beta}}{\q{\beta}}$ and
$\PairS{\WORK{\alpha}}{\FAIL}, \PairS{\WORK{\beta}}{\FAIL}$
are product locations. We omit the initial location and unreachable locations in the product construction.
From the construction of $\exttuples_{\dtloc}$'s, the functions $h_i$'s are as follows (we omit redundant labels such as $\emptyset$ and $\{\alpha,\beta\}$ which never appear in the PTA):
\begin{compactitem}
\item $h_0=\{\{ \alpha \} \mapsto y \le C_\alpha ,\{ \beta  \} \mapsto y \le W_\beta\}$;
\item $h_1=\{\{ \alpha \} \mapsto y \le C_\alpha ,\{ \beta  \} \mapsto W_\beta< y\}$;
\item $h_2=\{\{ \alpha \} \mapsto C_\alpha < y ,\{ \beta  \} \mapsto y \le W_\beta\}$;
\item $h_3=\{\{ \alpha \} \mapsto C_\alpha < y ,\{ \beta  \} \mapsto W_\beta< y\}$;
\item $h_4=\{\{ \beta \} \mapsto y \le C_\beta ,\{ \alpha  \} \mapsto y\le W_\alpha\}$;
\item $h_5=\{\{ \beta \} \mapsto y \le C_\beta ,\{ \alpha  \} \mapsto W_\alpha< y\}$;
\item $h_6=\{\{ \beta \} \mapsto C_\beta < y ,\{ \alpha  \} \mapsto y \le W_\alpha\}$;
\item $h_7=\{\{ \beta \} \mapsto C_\beta < y ,\{ \alpha  \} \mapsto W_\alpha< y\}$.
\end{compactitem}
The intuition is that the DTA accepts all infinite paths under the PTA such that the failing time for job $\gamma$ ($\gamma\in\{\alpha,\beta\}$) (the time within the consecutive $\gamma$'s) should be no greater than $C_\gamma$ and the waiting time for job $\gamma$ (the failing time plus the time spent on the last $\gamma$) should be no greater than $W_\gamma$.
%Here for the accepting mod  in DTA, we have
%\begin{align*}
%    &
%    \exttuples_{\dtloc_{\alpha}}
%        = \{
%            % \\
%            % &
%            h_0 = \left \{
%                \emptyset           \mapsto \true,
%                \{ \alpha \}        \mapsto ( y \le 500 ),
%                \{ \beta  \}        \mapsto ( y \le 200 ),
%                \{ \alpha, \beta \}  \mapsto \true
%            \right \},
%            \\
%            &
%            h_1 = \left \{
%                \emptyset           \mapsto \true,
%                \{ \alpha \}        \mapsto ( y \le 500 ),
%                \{ \beta  \}        \mapsto ( 200 < y ),
%                \{ \alpha, \beta \}  \mapsto \true
%            \right \},
%            \\
%            &
%            h_2 = \left \{
%                \emptyset           \mapsto \true,
%                \{ \alpha \}        \mapsto ( 500 < y  ),
%                \{ \beta  \}        \mapsto ( y \le 200 ),
%                \{ \alpha, \beta \}  \mapsto \true
%            \right \},
%            \\
%            &
%            h_3 = \left \{
%                \emptyset           \mapsto \true,
%                \{ \alpha \}        \mapsto ( 500 < y ),
%                \{ \beta  \}        \mapsto ( 200 < y ),
%                \{ \alpha, \beta \}  \mapsto \true
%            \right \}
%        \} .
%\end{align*}
%And a part of the product of Example~\ref{ex:pta} and Example~\ref{ex:dta} is depicted in Figure~\ref{fig:dta}.
%$
%    \PairS
%        {\WORK{\alpha}}
%        {\q{\alpha}},
%    \PairS
%        {\WORK{\beta}}
%        {\q{\beta}},
%    \PairS
%        {\WORK{\alpha}}
%        {\FAIL},
%    \mbox{ and }
%    \PairS
%        {\WORK{\beta}}
%        {\FAIL}
%$
%are locations.
\end{example}
% \vspace{-0.8em}

% \vspace{-0.8em}
%$\prob_\otimes$ satisfies that for any $\loc,\loc''\in\locs$, $\dtloc,\dtloc''\in\cstates$, $a\in\acts$, $r=\left(\dtloc, \lbfunc(\loc''), \phi', Y, \dtloc'\right)\in\rules$, $(X,\loc')\in 2^{\clocks_1}\times \locs$ and $Z\subseteq\clocks$, it holds that
%\[
%\prob_\otimes\left((\loc,\dtloc),(a,r)\right)(Z,(\loc'',\dtloc''))=\begin{cases}
%\prob\left(\loc,a\right)({Z}{\setminus}{Y},\loc'') & \mbox{if } \dtloc''=\dtloc' \\
%0 & \mbox{otherwise}
%\end{cases}\enskip.
%\]

Below we clarify the correspondence between $\pta,\dta$ and $\product{\pta}{\dta_\dtloc}$.
We first show the relationship between paths under $\pta$ and those under $\product{\pta}{\dta_\dtloc}$.
Informally, paths under $\product{\pta}{\dta_\dtloc}$ are just paths under $\pta$ extended with runs of $\dta$.

\smallskip\noindent
{\textbf{Transformation $\pfunc$ for Paths from $\pta$ into $\product{\pta}{\dta_\dtloc}$.}}
The transformation is defined as the function $\pfunc:\fnpaths{\pta}\cup\infpaths{\pta}\rightarrow \fnpaths{\product{\pta}{\dta_\dtloc}}\cup\infpaths{\product{\pta}{\dta_\dtloc}}$
which transform a finite or infinite path under $\pta$ into one under  $\product{\pta}{\dta_\dtloc}$.
For a finite path
$
%\[
\fnpath=(\loc_0,\nu_0)a_0\dots a_{n-1}(\loc_n,\nu_n)
%\]
$
under $\pta$ (note that $(\loc_0,\nu_0)=(\loc^*, \zero)$ by definition),
we define $\pfunc(\fnpath)$ to be the unique finite path
\begin{equation}\label{eq:trho}
\pfunc(\fnpath):=((\loc_0,\dtloc_0),\nu_0\cup\mu_0)a'_0\dots a'_{n-1}((\loc_n,\dtloc_n),\nu_n\cup\mu_n)
\end{equation}
under $\product{\pta}{\dta_\dtloc}$ such that the following conditions (\dag) hold:
\begin{compactitem}
\item $\dtatr{(\dtloc,\zero)}{\lbfunc(\loc^*)}{(q_0,\mu_0)}$ (note that $\mu_0=\zero$);
\item for all $0\le k< n$, if $a_k\in [0,\infty)$ then $a'_k=a_k$ and $\dtatr{(\dtloc_k,\mu_k)}{a_k}{\left(\dtloc_{k+1},\mu_{k+1}\right)}$;
\item for all $0\le k< n$, if $a_k\in\acts$ then $a'_k=(a_k,\xi_k)$ and $\dtatr{(\dtloc_k,\mu_k)}{\lbfunc(\loc_{k+1})}{\left(\dtloc_{k+1},\mu_{k+1}\right)}$ where $\xi_k$ is the unique function such that for each symbol $b\in\alphabet$, $\xi_k(b)$ is the unique clock constraint appearing in a rule emitting from $q_k$ and with symbol $b$ such that $\mu_k\models\xi_k(b)$.
\end{compactitem}
Likewise, for an infinite path $\infpath=(\loc_0,\nu_0)a_0(\loc_1,\nu_1)a_1\dots$
under $\pta$, we define $\pfunc(\infpath)$ to be the unique infinite path
\begin{equation}\label{eq:trinfpath}
\pfunc(\infpath):=((\loc_0,\dtloc_0),\nu_0\cup\mu_0)a'_0((\loc_1,\dtloc_1),\nu_1\cup\mu_1)a'_1\dots
\end{equation}
under $\product{\pta}{\dta_\dtloc}$ such that the three conditions in (\dag) hold for all $k\in\Nset_0$ instead of all $0\le k< n$. From the determinism and totality of $\dta$, it is straightforward to prove the following result.
%The following lemma shows that $\pfunc$ is a bijection and preserves zenoness.
\begin{lemma}\label{lemm:pfuncbij}
The function $\pfunc$ is a bijection. Moreover, for any infinite path $\infpath$ under $\pta$, $\infpath$ is non-zeno iff $\pfunc(\infpath)$ is non-zeno.
\end{lemma}
%\begin{proof}
%The first claim follows directly from the determinism and totality of DTAs.
%The second claim follows from the fact that $\pfunc$ preserves time elapses in the transformation.
%\end{proof}

%between $\fnpaths{\pta,\sigma}\cup\infpaths{\pta,\sigma}$ and $\fnpaths{\product{\pta}{\dta_q},\sfunc\left(\sigma\right)}\cup\infpaths{\product{\pta}{\dta_\dtloc},\sfunc\left(\sigma\right)}$.

Below we also show the correspondence on schedulers before and after the product construction.

\smallskip
\noindent{\textbf{Transformation $\sfunc$ for Schedulers from $\pta$ into $\product{\pta}{\dta_\dtloc}$.}}
We define the function $\sfunc$ from the set of schedulers under $\pta$ into the set of schedulers under $\product{\pta}{\dta_\dtloc}$ as follows: for any scheduler $\sigma$ for $\pta$, $\sfunc(\sigma)$ (for $\product{\pta}{\dta_\dtloc}$) is defined such that for any finite path $\fnpath$ under $\pta$ where $\fnpath=(\loc_0,\nu_0)a_0\dots a_{n-1}(\loc_n,\nu_n)$ and $\pfunc(\rho)$ given as in (\ref{eq:trho}),
\[
\sfunc(\sigma)(\pfunc(\fnpath)):=
\begin{cases}
\sigma(\fnpath) & \mbox{if }n\mbox{ is even} \\
(\sigma(\fnpath),\lambda(\rho)) & \mbox{if }n\mbox{ is odd}
\end{cases}
\]
where $\lambda(\rho)$ is
the unique function such that for each symbol $b\in\alphabet$, $\lambda(\rho)(b)$ is the clock constraint in the unique rule emitting from $q_n$ and with symbol $b$ such that $\mu_n\models\lambda(\rho)(b)$.
Note that the well-definedness of $\sfunc$ follows from Lemma~\ref{lemm:pfuncbij}.

From Lemma~\ref{lemm:pfuncbij}, the product construction, the determinism and totality of $\rules$, one can prove directly the following lemma.
%\vspace{-0.8em}
\begin{lemma}\label{lemm:sfuncbij}
The function $\sfunc$ is a bijection.
\end{lemma}
%\vspace{-0.8em}
Now we prove the relationship between infinite paths accepted by a DTRA before product construction and infinite paths satisfying certain Rabin condition.

We introduce more notations.
First, we lift the function $\pfunc$ to all subsets of paths in the standard fashion: for all subsets $A\subseteq \fnpaths{\pta}\cup\infpaths{\pta}$, $\pfunc(A):=\{\pfunc(\omega)\mid \omega\in A\}$.
Then for an infinite path $\infpath$ under $\product{\pta}{\dta_\dtloc}$ in the form
(\ref{eq:trinfpath}), we define
%%%%%
% \vspace{-0.8em}
%\begin{definition}[Traces]
%Let
%$ \pfunc(\infpath) =
%    ((\loc_0,\dtloc_0),\nu_0\cup\mu_0)
%    a'_0
%    ((\loc_1,\dtloc_1),\nu_1\cup\mu_1)
%    a'_1
%    \dots
%$
% and according to definition
% $Lab((\loc_i,\dtloc_i)�� = \left\{ \dtloc_i \right\}$.
% Instead of using $\{ q_0 \}, \{ q_1 \}, \dots $,
the \emph{trace} of $ \infpath  $ as an infinite word over $\cstates$ by
$\trace{  \infpath  } := \dtloc_0 \dtloc_1 \dots $ .
Finally, for any scheduler $\sigma$ for $\product{\pta}{\dta_\dtloc}$,
we define the set $\rabinp{\sigma}$ by
%\end{definition}
%%%%%
% \vspace{-0.8em}
% \begin{proposition}\label{prop:psfunc} $\pfunc\left(\acc{\pta,\sigma}{\dta,\dtloc,F}\right)=\omgpaths{\product{\pta}{\dta_\dtloc},\sfunc\left(\sigma\right)}{\locs\times F}$ for any scheduler $\sigma$, initial mode $q$ and $F\subseteq \cstates$. %or in Definition~\ref{def:infacc}.
% \end{proposition}
% \begin{proof}
% Directly from the product construction and Definition~\ref{def:fnacc}.\qed
% \end{proof}
%\noindent{\textbf{Verifying Limit Rabin Properties.}}
%Paths in $\product{\pta}{\dta_\dtloc}$ that $\pta$ is accepted by $\dta$ is
% д�ɶ�������ͨ�� PTA �ϻ��dz˻��Ϻ�
% $$
%     \AcceptCxAqsF = \left \{
%         \infpath \in \infpaths{\product{\pta}{\dta_\dtloc},\sigma} \mid
%         \infset{
%             \trace{
%                 \infpath
%             }
%         }
%         \mbox{ is Rabin accepting by } \rabin
%     \right\}
% $$
%\begin{small}
$$
    \rabinp{\sigma}:=\left \{
        \infpath \in \infpaths{\product{\pta}{\dta_\dtloc},\sigma} \mid
        \accept{
            \infset{
                \trace{
                    \infpath
                }
            }
        }   {
            \rabin
        }
    \right\}.
$$
%\end{small}
Intuitively, $\rabinp{\sigma}$ is the set of infinite paths under $\product{\pta}{\dta_\dtloc}$ that meet the Rabin condition $\rabin$ from $\dta$.
The following proposition clarifies the role of $\rabinp{\sigma}$.
%and $\AcceptCxAqsF$ is an limit LT Property \cite[Notation10.121]{DBLP:books/daglib/0020348}.
%
% \vspace{-0.8em}
\begin{proposition}\label{prop:psfunc}
For any scheduler $\sigma$ for $\pta$ and any initial mode $q$ for $\dta$, we have $\TLang = \TAcc.$
\end{proposition}
%\begin{proof}
%% This proof is trivial.
%By definition, the set $\LangCsAqF$ equals
%% $$
%%     \LangCsAqF = \left \{
%%         \infpath \in \infpaths{\pta,\sigma} \mid
%%         \infset{
%%             \traj{
%%                 \run{\dta}{\iconfig}{\lbfunc(\infpath)}
%%             }
%%         }
%%         \mbox{ is Rabin accepting by } \rabin
%%     \right\},
%% $$
%%\begin{small}
%$$
%    \left \{
%        \infpath \in \infpaths{\pta,\sigma} \mid
%        \accept{
%            \infset{
%                \traj{
%                    \xi_\pi%\run{\dta}{\iconfig}{\lbfunc(\infpath)}
%                }
%            }
%        }   {
%            \rabin
%        }
%    \right\}
%$$
%%\end{small}
%where $\xi_\pi$ is the unique run of $\dta$ on $\lbfunc(\pi)$ with initial configuration $(\dtloc^*,\zero)$ for which
%$\dtloc^*$ is the unique location such that $\tran{(q,\zero)}{\lbfunc(\loc^*)}{(\dtloc^*,\zero)}$.
%Let $\infpath = (\loc_0,\nu_0) a_0 (\loc_1,\nu_1) a_1 \dots $ be any infinite path.
%By the definition of $\pfunc$ we have
%$$
%    \pfunc( \infpath ) =
%        ((\loc_0,\dtloc_0),\nu_0\cup\mu_0)
%        a'_0
%        ((\loc_1,\dtloc_1),\nu_1\cup\mu_1)
%        a'_1
%        \dots
%$$
%in the form (\ref{eq:trinfpath}) with $\xi_\pi=\{(\dtloc_n,\mu_n,a_{n})\}_{n\in\Nset_0}$ being the unique run
%on $\lbfunc(\infpath)=a_0a_1\dots$ .
%% �Ҿ��ò���Ҫ����֤ T��pi) �� theta��sigma�� ���� д����
%%$$
%%    \run{\dta}{\iconfig}{\lbfunc(\infpath)}
%%        = \{(\dtloc_n,\mu_n,\lbfunc(\infpath)_{n})\}_{n\in\Nset_0}.
%%$$
%Then it is obvious that
%$$\trace{ \pfunc( \infpath ) }
%    = q_0 q_1 \dots
%    = \traj{
%        \run
%            {\dta}{\iconfig}
%            {\lbfunc(\infpath)}
%    }.
%$$
%It follows that
%$\infset {
%    \trace {
%        \pfunc( \infpath )
%    }
%}$
%is Rabin accepting by $\rabin$ iff
%$
%    \infset{
%        \traj {
%            \run
%                {\dta}{\iconfig}
%                {\lbfunc(\infpath)}
%        }
%    }
%$
%is Rabin accepting by $\rabin$.
%\end{proof}

Finally, we demonstrate the relationship between acceptance probabilities before product construction and Rabin(-accepting) probabilities after product construction.
We also clarify the probability of zenoness before and after the product construction.
The proof follows standard argument from measure theory.

% \vspace{-0.8em}
\begin{proposition}\label{thm:main}
For any scheduler $\sigma$ for $\pta$ and mode $q$, the followings hold:
\begin{compactitem}
\item
{\small $
    \pr
        {\dtloc}
        {\sigma}
        =
            \probm
                ^{\pta,\sigma}
                \left(
                % \acc{\pta,\sigma}{\dta,q,F}
                    \LangCsAqF
                \right)
        =
            \probm
                ^{\product{\pta}{\dta_\dtloc},\theta(\sigma)}
                \left(
                    % \omgpaths{\product{\pta}{\dta_\dtloc},\theta\left(\sigma\right)}{\locs\times F}
                    \TAcc
                \right)
    ;
$}
\item
%Moreover,
{\small$
    \probm
        ^{\pta,\sigma}
        \left( \{
                \infpath \mid \infpath \mbox{ is zeno}
            \}
        \right)
    =
    \probm
        ^{\product{\pta}{\dta_\dtloc},\theta(\sigma)}
        \left( \{
                \infpath' \mid \infpath' \mbox{ is zeno}
            \}
        \right).
$}
\end{compactitem}
\end{proposition}
%\vspace{-0.8em}
%\begin{proof}
%Define the probability measure $\probm'$ by: $\probm'(A)=\probm^{\product{\pta}{\dta_\dtloc},\sfunc\left(\sigma\right)}(\pfunc(A))$ for $A\in\mathcal{F}^{\pta,\sigma}$. We show that $\probm'=\probm^{\pta,\sigma}$. By \cite[Theorem 3.3]{PBMeasure}, it suffices to consider cylinder sets as they form a pi-system (cf. \cite[Page 43]{PBMeasure}).
%Let $\fnpath=(\loc_0,\nu_0)a_0\dots a_{n-1}(\loc_n,\nu_n)$ be any finite path under $\pta$.
%By definition, we have that
%\begin{align*}
%    \probm^{\pta,\sigma}(\cyl(\fnpath))
%        & =
%        \probm^{\product{\pta}{\dta_\dtloc}, \sfunc\left(\sigma\right)}(\cyl(\pfunc(\fnpath)))
%        \\
%        & =
%        \probm^{\product{\pta}{\dta_\dtloc},\sfunc\left(\sigma\right)}(\pfunc(\cyl(\fnpath)))
%        \\
%        & =
%        \probm'(\cyl(\fnpath))\enskip.
%\end{align*}
%The first equality comes from the fact that both versions of product construction preserves transition probabilities. The second equality is due to $\cyl(\pfunc(\fnpath))= \pfunc(\cyl(\fnpath))$.
%The final equality follows from the definition.
%Hence $\probm^{\pta,\sigma}=\probm'$.
%Then the first claim follows from Proposition~\ref{prop:psfunc} and the second claim follows from Lemma~\ref{lemm:pfuncbij}.
%\end{proof}
A side result from Proposition~\ref{thm:main} says that $\sfunc$ preserves time-divergence for schedulers before and after product construction.
From Proposition~\ref{thm:main} and Lemma~\ref{lemm:sfuncbij}, one immediately obtains the following result which transforms the {\sc PTA-DTRA} problem into Rabin(-accepting) probabilities under the product PTA.

% \vspace{-0.8em}
\begin{corollary}\label{crly:opt}
For any initial mode $q$,
% there exists an $\rabin_* \subseteq \locs_\otimes$
% s.t. $\rabin_*$ is a union of several maximal end
% components that satisfy $\rabin$.
% \begin{align*}
$
    \opt_\sigma\pr{\dtloc}{\sigma}
        =
            \opt_{\sigma'}
            \probm^{\product{\pta}{\dta_\dtloc},\sigma'}\left(
%                \Accept
%                    {\product{\pta}{\dta_\dtloc},\sigma'}
            \rabinp{\sigma'}
                    % {\rabin}
            \right)
        % \\
        % & =
        %     \opt_{\sigma''}
        %     \probm^{\product{\pta}{\dta_\dtloc},\sigma''}\left(
        %         \omgpaths
        %             {\product{\pta}{\dta_\dtloc},\sigma''}
        %             {\rabin_*}
        %     \right)
        % \\
        % & =
        %     \opt_{\sigma'''}
        %     \probm^{\product{\pta}{\dta_\dtloc},\sigma'''}\left(
        %         \omgpaths
        %             {\product{\pta}{\dta_\dtloc},\sigma'''}
        %             {\rabin_*}
        %     \right),
% \end{align*}
$
where $\opt$ refers to either $\inf$ (infimum) or $\sup$ (supremum),
$\sigma$ (resp. $\sigma'$) range over all time-divergent schedulers
for $\pta$ (resp. $\product{\pta}{\dta_\dtloc}$).
% where $\opt$ refers to either $\inf$ (infimum) or $\sup$ (supremum),
% $\sigma$ (resp. $\sigma'$) range over all time-divergent schedulers
% for $\pta$ (resp. $\product{\pta}{\dta_\dtloc}$) and $\sigma''$ (resp.
% $\sigma'''$) range over all time-divergent (resp. all time-divergent and
% time-convergent) schedulers for $\product{\pta}{\dta_\dtloc}$. $\rabin_*$ can be
% resolved by an MEC algorithm and $\opt_\sigma\pr{\dtloc,\rabin}{\sigma}$
% can be calculated by a reachability analysis.
\end{corollary}

\noindent{\bf Solving Rabin Probabilities.} We follow the approach in~\cite{DBLP:conf/qest/Sproston11}
to solve Rabin probabilities over PTAs. Below we briefly describe the approach.
The approach can be divided into two steps.
The first step is to ensure time-divergence.
This is achieved by (i) making a copy for every location in the PTA, (ii) enforcing a transition
from every location to its copy to happen after $1$ time-unit elapses, (iii)
enforcing a transition from every copy location back to the original one immediately with no time-delay,
and (iv) putting a special label \textit{tick} in every copy.
Then time-divergence is guaranteed by adding the label  \textit{tick} to the Rabin condition.
%We only deal with
%paths that satisfy $ \square \Diamond tick $ (i.e. \textit{tick} is satisfied
%infinitely many times).
The second step is to transform the problem into limit Rabin properties over MDPs~\cite[Theorem 10.127]{DBLP:books/daglib/0020348}.
This step constructs an MDP $\Region{\product{\pta}{\dta_\dtloc}}$ from the
PTA $ \product{\pta}{\dta_\dtloc} $ through a \emph{region-graph} construction
so that the problem is reduced to solving limit Rabin properties over $\Region{\product{\pta}{\dta_\dtloc}}$.
\emph{Regions} are finitely-many equivalence classes of clock valuations that serve as a finite abstraction which capture exactly reachability behaviours over timed transitions (cf.~\cite{DBLP:journals/tcs/AlurD94}).
Then standard methods based on \emph{maximal end components} (MECs) are applied to $\Region{\product{\pta}{\dta_\dtloc}}$.
In detail, the algorithm computes the reachability probability to MECs that satisfy the Rabin acceptance condition.
In order to guarantee time-divergence, the algorithm only picks up MECs with at least one location that has a \textit{tick} label.
Based on this approach, our result leads to an algorithm for solving the problem PTA-DTRA.

%let $F_*$ be the union of those MECs. Then, we turn to resolve the probability
%reachability to $F_*$.
%
%\begin{lemma}{Time Complexity of Verifying Limit Rabin Properties(
%    \cite[Theorem 10.127]{DBLP:books/daglib/0020348}
%)}\label{lemm:principle}
%Let $M$ be a finite MDP and $P$ be a limit LT propery specified by a Rabin condition:
%$$
%\bigvee_{1 \le i \le n} \left(
%    \Diamond \square \lnot H_i
%    \land
%    \square \Diamond K_i
%\right)
%$$
%Then: the values
%$
%\opt_\sigma \probm^{M,\sigma} \left(
%    s \models P
%\right)
%$
%can be computed in time \\
%$
%\mathcal{O} \left(
%    \mbox{poly} \left(
%        \mbox{size} \left(
%            M
%        \right)
%    \right)
%    \cdot
%    k
%\right)
%$
%where $\opt$ refers to either $\inf$ (infimum) or $\sup$ (supremum).
%\end{lemma}

Note that in $\product{\pta}{\dta_\dtloc}$,
although the size of $\acts_\otimes$
may be exponential due to possible exponential blow-up from $\exttuples_{\dtloc}$,
one easily sees that $ | \locs_\otimes | $ is $ |\locs| \cdot |\cstates| $ and
$ | \clocks_\otimes | = | \clocks | + | \dtclocks| $.
Hence, the size of $ \Region{\product{\pta}{\dta_\dtloc}} $ is still exponential in
the sizes of $\pta$ and $\dta$.
%$ |\locs| \cdot |\cstates| $,
%while the number of transitions is exponential,
% that the computation of $F_*$ can be done in polynomial time in the size of
% $ \Region{\product{\pta}{\dta_\dtloc}} $,
It follows that $\opt_\sigma\pr{\dtloc}{\sigma}$ can be calculated in exponential time
from the MEC-based algorithm illustrated in~\cite[Theorem 10.127]{DBLP:books/daglib/0020348}, as is demonstrated by
the following proposition.

\begin{proposition}\label{prop:exptime}
The problem {\sc PTA-DTRA} is in EXPTIME in the size of the input PTA and DTRA.
\end{proposition}

It is proved in \cite{LaroussinieS07} that the reachablity-probability problem for arbitrary PTAs is \emph{EXPTIME}-complete.
Since Rabin acceptance condition subsumes reachability, one obtains that the problem PTA-DTRA is EXPTIME-hard (cf. Appendix~\ref{app:hardness} for details).
Thus we obtain the main result of this section which settles the computational complexity of the problem PTA-DTRA.


\begin{theorem}
The PTA-DTRA problem is EXPTIME-complete.
\end{theorem}
%
%\vspace{-0.8em}
%
%The reward structure $(\rcum',\rinst')$ for $\pta$ is defined as follows:
%\begin{compactitem}
%\item $\rcum'(\loc,\dtloc):=\rcum(\loc)$ for any $(\loc,\dtloc)\in \locs_\otimes$;
%\item $\rinst'((\loc,\dtloc),(a,h)):=\rinst(\loc,a)$ for any $(\loc,\dtloc)\in \locs_\otimes$ and $(a,h)\in \acts_\otimes$.
%\end{compactitem}
\begin{remark}
The main novelty for our product construction is that by adopting extended actions (i.e. $\exttuples_{\dtloc}$) and integrating them into enabling condition and probabilistic transition function, the product PTA can know which rule to use from the DTA upon any symbol to be read. This solves the problem that probabilistic jumps can lead to different locations, causing the usage of different rules from the DTA. Moreover, our product construction ensures EXPTIME-completeness of the problem.
\end{remark}
